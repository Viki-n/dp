\chapter{Výsledky}

V~této kapitole popíšeme chování naší implementace stromů na několika třídách přístupových posloupností.

\section{Náhodná přístupová posloupnost}

Třída náhodných přístupových posloupností vypadá tak, že pro danou velikost
strom vyrobíme náhodnou posloupnost přirozených čísel z~intervalu $[0,n)$.
Náhodná čísla generujeme pomocí pseudonáhodného generátoru {\tt xoshiro256**},
který představili \citet{xoshiro}. Pro měření jsme zvolili velikosti stromů od $10$ do $1\cdot
10^8$, kde každá další testovaná velikost byla $\sqrt[7]{10}$-krát větší, než ta předchozí. Takovýto krok
jsme zvolili proto, že s~ním v~celém rozsahu měříme celkem 50 různých velikostí
stromu, tedy dosáhneme rozumného kompromisu mez počtem vzorků a celkovou
dobou měření. 

Délka posloupnosti byla vždy $\max(1\cdot10^7, 10\cdot n)$.
Takové číslo jsme zvolili proto, abychom snížili pravděpodobnost, že dostaneme
náhodnou posloupnost, jejíž čas běhu a počet dotyků na operaci budou daleko od
příslušné střední hodnoty, abychom i pro větší $n$ alespoň jednou navštívili
téměř všechny prvky a aby
nepřesnost měření času (která může dosahovat až řádově desítek milisekund)
nehrála významnou roli v~celkovém času vykonávání posloupnosti. 

\def\graphfigure#1#2{
\begin{figure}[h!]
\centering
 \makebox[\textwidth][c]{\includegraphics{graphs/#1}}
\caption{#2}
\label{obr:#1}
\end{figure}
}

\graphfigure{touch_r}{Doteky vrcholů -- náhodná přístupová posloupnost.}


Při vlastním měření jsme ale neměřili dvě nejvyšší hodnoty $n$ pro tango strom kvůli vysoké časové náročnosti měření.

Na základě teorie bychom čekali, že počet doteků na operaci u~multisplay, splay a červenočerného stromu poroste logaritmicky s~počtem vrcholů, u~tango stromu ještě s~faktorem $\log\log$ navíc, což měření potvrzují, viz obrázek \ref{obr:touch_r}.
\let\oldlog\log
\def\log{\oldlog_2}

Na grafu vidíme, že červenočerný strom potřebuje konzistentně přibližně $(\log
n) - 1/2$. $\log n$ jsme do grafu přidali pro porovnání. To zní podezřele --
pro červenočerný strom je střední hodnota počtu doteků na operaci z~náhodné
přístupové posloupnosti prostě průměrná hloubka vrcholu (kde kořen má hloubku
1). Dokonale vyvážený strom na $n$ vrcholech má průměrnou hloubku vrcholu
vrcholu $(\log n) - 1 + o(1)$. My jsme ale v~minulé kapitole řekli, že náš
červenočerný strom je tak nevyvážený, jak je jen možné.

Jak je tedy možné, že průměrná hloubka vrcholu ve velmi nevyváženém
červenočerném stromu je jen o~aditivní $1/2$ větší než v~dokonale vyváženém
stromu? Odpověď na tuto otázku je v~nejasné definici velmi nevyváženého stromu.
Náš strom vypadá tak, že na jeho pravé páteři se střídají černé a červené
vrcholy, ale celý zbytek stromu krom této cesty a vrcholů s~ní bezprostředně
sousedících je stále černý\footnote{Toto tvrzení dokážeme později.}. Proto je i vyvážený, jak to jen jde -- kvůli
invariantu červenočerného stromu je každý čistě černý podstrom dokonale
vyvážený.

Náš strom je
tedy extrémně nevyvážený v~tom smyslu, že dosahuje co nejvyšší možné maximální
hloubky při daném počtu vrcholů. Také je extrémně nevyvážený stran poměru
velikosti pravého a levého podstromu kořene (v~závislosti na přesné velikosti
experimentálně určeného jako $2:3$ až $1:4$). Protože však současně jsou
podstromy všech vrcholů krom těch ležících na pravé páteři dokonale
vyvážené, stran průměrné hloubky vrcholu je takto vybudovaný strom poměrně
blízko dokonale vyváženému stromu.


Splay strom potom potřebuje zhruba $1.28\cdot \log n$ doteku na přístup, multisplay strom
dokonce $3.35\cdot \log n$. U~tango stromu jsme na našem vzorku naměřili
v~průměru $4.88\cdot \log n$ doteků, což ale není příliš zajímavé -- u~ostatních
stromů očekáváme, že bude počet doteků na operaci dělený logaritmem velikosti stromu konvergovat
k~nějaké konstantě, u~tango stromu toto očekávání nemáme. Nejvyšší zaznamenaný
počet dotyků pro tango strom byl v~testu pro $n \cong 5.18\cdot 10^7$. V~tomto
testu jsme naměřili asi $6.81\cdot \log n$ doteků na operaci. Pro srovnání,
$\log\log 5.18\cdot 10^7 \cong 4.68$. 

Dále si všimneme mírného rozvlnění křivky tango stromu kolem $n=5\cdot 10^3$ a
znovu kolem $n=1\cdot 10^6$. Tento efekt se ale výrazněji projeví při
sekvenčních přístupech, které budeme diskutovat v~následující sekci, s~jeho
vysvětlením tedy počkáme tam.

\graphfigure{time_r}{Průměrný čas na operaci -- náhodná přístupová posloupnost.}

Na obrázku \ref{obr:time_r} vidíme naměřené časy. Proti obrázku \ref{obr:touch_r}
se křivka tango stromu ještě trochu vzdálila ostatním. Dále vidíme zlom všech
křivek pro $n$ mezi $1\cdot10^6$ a $1\cdot 10^7$. Vzhledem k~tomu, že použitý
procesor má $128\,\operatorname{MB}$ L3 cache a jeden vrchol  červenočerného či
splay stromu zabírá v~paměti 24 bytů, přibližně při $3\cdot 10^6$ vrcholech se již
strom nevejde do cache současně s~$40\,\operatorname{MB}$, které zabírá načtená
část přístupové posloupnosti. Přibližně při $5\cdot10^6$ vrcholech se pak již strom nevejde do cache ani sám.
Jeden vrchol multisplay stromu a tango stromu pak zabírá 32 bytů a výše uvedené pro ně platí obdobně. 

\graphfigure{ratio_r}{Průměrný čas na dotyk -- náhodná přístupová posloupnost.}

\subsection{Červenočerný strom a cache}\label{sec:rb_and_cache}

Překvapivé je, že kolem $n=2\cdot10^6$ se splay strom stává rychlejší než
červenočerný strom. Ve skutečnosti se podle grafu na obrázku \ref{obr:ratio_r}
průměrný čas na dotek vrcholu v~červenočerném stromu propadá až téměř na úroveň tango
stromu -- to je obzvláště překvapivé proto, že červenočerný strom vrcholy pouze
prochází a nijak strom nemodifikuje, kdežto všechny ostatní stromy vrcholy
modifikují a navíc jimi mohou projít během jedné operace i vícekrát, což my
ovšem ale stále počítáme jako jeden dotyk.

Toto chování má své vysvětlení, předtím, než k~němu přistoupíme ale musíme
udělat krátkou odbočku, v~níž vysvětlíme, jak funguje cache procesoru.

\subsubsection{Chování cache}

Procesor má obvykle tři úrovně cache, nejmenší L1, o~něco větší L2 a největší
L3. To pro nás ale teď nebude důležité -- omezíme se pouze na jednu úroveň.
Cache je rozdělená na jednotlivé řádky, v~dnešních procesorech typicky 64 bytů
velké. Pokud procesor chce přistoupit na nějakou adresu, blok paměti
o~velikosti jednoho cachového řádku zarovnaný na velikost cachového
řádku\footnote{To znamená, že adresa začátku tohoto bloku je dělitelná délkou
cachového řádku.} obsahující tuto adresu je načten do cache. Může se ale stát,
že v~cachi už není volné místo. V~ideálním světě by v~takovém případě cache
vybrala z~řádků, které obsahuje, ten, ke kterému procesor nejdéle nepřistoupil,
a ten by odstranila. Tomuto přístupu obecně říkáme \emph{LRU\footnote{Z
anglického \emph{least recently used}.} cache}.

Pro to bychom ale potřebovali, aby libovolný řádek cache mohl obsahovat
libovolný blok paměti. Takovou cache bychom nazvali \emph{plně asociativní}.
Potom by ale bylo velmi obtížné zjistit, kde v~cachi je daný blok paměti
uložený. Proto jsou v~praxi cache \emph{$k$-cestně asociativní}. To znamená, že
každý blok může být uložen pouze na $k$ různých místech v~cachi. Která místa to
budou je pak určeno několika nejnižšími bity adresy bloku. Konkrétně pokud máme
cache o~$C$ bytech, uložení bloku určuje $\log (C/k)$ nejnižších bitů jeho
adresy.

To znamená, že pokud jsou dva bloky paměti uložené na adresách, které dávají
stejný zbytek po dělení dostatečně velkou mocninou dvojky, mají oba na výběr
pouze $k$ míst v~cachi, kam mohou být uloženy. Navíc se jedná o~těch samých $k$
míst. Tomu říkáme, že tyto bloky, případně adresy, \emph{aliasují}.

Kdybychom potřebovali pracovat s~větším počtem aliasujících bloků, přestává být
podstatný celkový počet řádků, které se do cache vejdou -- o~tom, jak úspěšná
cache bude, rozhoduje pouze hodnota $k$.

\subsubsection{Cache a dokonale vyvážené stromy}\label{sec:cache_and_balanced}

Nyní se vrátíme k~červenočerným stromům. Již dříve jsme řekli, že náš postup stavby červenočerného stromu vede k~tomu, že podstrom každého vrcholu krom vrcholů na pravé páteři je dokonale vyvážený. Proto můžeme pro zjednodušení problém vysvětlit na dokonale vyváženém stromu. Pro strom, který dostaneme z~naší konstrukce červenočerného stromu budou závěry, které uděláme, platit podobně. Jediná důležitá věc, které si ještě musíme všimnout, je, že náš způsob alokace paměti spolu s~naším způsobem stavby stromu vede k~tomu, že vrcholy jsou v~paměti uloženy jeden za druhým v~rostoucím pořadí.

Mějme tedy dokonale vyvážený strom nad množinou klíčů $[2^k-1]$. Všimneme si že v~listech jsou uložena právě lichá čísla. V~jejich rodičích jsou uložena právě čísla, která dávají zbytek 2 po dělení 4. V~jejich rodičích pak čísla, která dávají zbytek 4 po dělení 8. Takto bychom mohli pokračovat dál až k~tomu, že v~kořeni je právě číslo, které dává zbytek $2^{k-1}$ při dělení $2^k$ (mimo jiné proto, že v~kořeni je přesně číslo $2^{k-1}$).

Na to se můžeme ale podívat i opačně, shora dolů. Nahlédneme, že v~kořeni je právě číslo dělitelné $2^{k-1}$. V~kořeni a jeho synech jsou pak dohromady právě všechna čísla dělitelná $2^{k-2}$. Podobně můžeme pokračovat dál -- všimneme si, že v~prvních $h$ hladinách jsou uložena právě všechna čísle, která jsou dělitelná $2^{k-h}$. A~jsou-li tedy vrcholy uložené za sebou v~rostoucím pořadí jejich klíčů, a navíc předpokládáme, že množství bytů v~paměti, které jeden vrchol zabírá, je dělitelné 8 (tento předpoklad splňují všechny naše implementace stromů), potom adresy všech vrcholů z~prvních $h$ hladin dávají po dělení $2^{k-h+3}$ stejný zbytek.

Nyní se zamyslíme, jaké chování cache bychom si přáli při vykonávání náhodné přístupové posloupnosti nad dokonale vyváženým binárním vyhledávacím stromem. Pravděpodobnost návštěvy vrcholu $v$ při daném přístupu je přesně rovna velikosti jeho podstromu dělené počtem vrcholu ve stromu. To znamená, že tato pravděpodobnost je klesající funkce hloubky tohoto vrcholu. My bychom si přáli držet si v~cachi (samozřejmě za podmínky, že máme příliš malou cache, než aby se nám do ní vešel celý strom) právě vrcholy, které mají pravděpodobnost přístupu alespoň $p$ pro vhodně zvolené $p$. Rozmyslíme si, že (opět pro dostatečně velký strom) není možné, že bychom do jednoho řádku cache uložili více zajímavých vrcholů (například v~našem případě -- vrcholy stromu jsou 24 bytové, řádky cache 64 bytové. I~kdybychom do jednoho řádku uložili 3 vrcholy, oba sousedi libovolného vrcholu $v$ s~nízkou hloubkou jsou listy, a jejich sousedé různí od $v$ jsou jejich rodiče). Proto prostě chceme do cache uložit prvních $h$ hladin stromu pro vhodně zvolené $h$.

Jak jsme ale již ukázali, pokud má strom celkem $\ell$ hladin, prvních $h$
hladin obsahuje vrcholy, jejichž adresy jsou kongruentní modulo $2^{\ell - h}$.
Pokud naše cache má kapacitu $2^c$ bytů a je $k$-asociativní, budou tyto adresy
rozděleny pouze do $\max\left(1, 2^{c+h-\ell - 3}/k\right)$ tříd
ekvivalence aliasování, a současně jich tedy bude moct být uloženo nejvýše
$\max\left(k, 2^{c+h-\ell - 3}\right).$

Pokud do tohoto vzorce dosadíme náš hardware a náš největší měřený případ,
dostaneme dostaneme následující informace: Naše L3 cache má $128
\,\operatorname{MB}$. Jeden cachový řádek má 64 bytů. Naše cache má tedy
$2^{27}$ bytů a $2^{21}$ řádků. Proto bychom do ní rádi uložili prvních 21 hladin stromu. Náš
největší měřený případ měl $1\cdot 10^8 > 2 ^ {26}$ vrcholů. Z~těchto ale
můžeme současně uložit jen $2^{27 + 21 - 26 - 3} = 2^{19}$, tedy jednu čtvrtinu
toho, co bychom mohli s~plně asociativní cachí. Co hůře, všimneme si, že dokud
$2^h \geq 4k$, poměr mezi částí stromu, kterou bychom chtěli  uložit, a částí,
kterou se nám skutečně uložit povede, zůstává konstantní $1/4$. Jinými slovy,
nemůžeme doufat, že do cache tedy uložíme prostě o~2 hladiny méně, než jsme
doufali -- pokud bychom se pokusili uložit $2^{19}$ vrcholů z~prvních 19
hladin, podaří se nám ve skutečnosti uložit pouze $2^17$ vrcholů. To znamená,
že z~každé hladiny, která obsahuje více než $4k$ vrcholů, se nám podaří uložit
pouze čtvrtinu (stále se jedná o~optimistický odhad, předpokládáme, že hladiny
mezi sebou třídy ekvivalence aliasování nesdílí, dostatečně blízko ke kořeni
však tento předpoklad nemusí platit). 

V~našem případě se $k$ rovná $16$. To znamená, že můžeme plně uložit prvních pět hladin,
z~šesté uložíme polovinu, ze všech dalších pak už jen čtvrtinu až po 23., což je
poslední hladina, z~níž se jedna čtvrtina do cache vejde, i kdyby v~ní nic
jiného než tato hladina nebylo. Na dalších úrovních pak shora odhadneme
pravděpodobnost přítomnosti daného vrcholu v~cachi popořadě jako $1/8$, $1/16$
a tak dále. To znamená, že střední hodnota počtu nenacachovaných řádků, které
musíme načíst při přístupu k~náhodnému listu stromu, je díky linearitě střední
hodnoty $$\sum_{h=1}^{26} \operatorname{P}[\text{Vrchol na $h$-té hladině není v
cachi}] > 5 \cdot 0 +  \frac 12 + 17\cdot \frac34 + \frac78 + \frac{15}{16} + \frac{31}{32} = 15\frac1{32},$$ což je výrazně horší než pět přístupů mimo cache, ve které bychom mohli doufat, kdyby naše cache byla plně asociativní.   

V~tomto výpočtu jsme navíc
zanedbali další faktory, které nám mohou škodit -- do cache se musí vejít i
jiné věci, než prvních $h$ hladin stromu, jako třeba náš program, naše poslední
cesty stromem mimo prvních $h$ hladin, nebo aktuální kus přístupové
posloupnosti. Dále je potřeba si uvědomit, že podobný problém nastane na všech
úrovních cache.

\subsubsection{Praktické chování červenočerných stromů}

Tím bychom tedy měli vyřešené chování dokonale vyváženého stromu a můžeme se
vrátit k~červenočernému stromu. Experimentálně jsme ověřili, že naší
implementací červenočerného stromu mají pro $n=1\cdot 10^7$  adresy vrcholů
v~prvních 8 hladinách stromu, tedy 255 vrcholů nejblíže ke kořeni, všechny
stejných 17 nejnižších bitů. Více než polovina z~nich se shodne dokonce na 19
bitech. Pro $n=1\cdot10^8$ mají adresy 255 nejvyšších vrcholů dokonce 21
společných nejnižších bitů.   

Dále jsem ověřili, zda problém skutečně nastává. Připomeneme, že naše struktury stavíme tak, že nejprve naalokujeme pole tak velké, aby se nám do něj vešel celý strom, jehož začátek je zarovnaný na začátek stránky paměti, a poté vždy, když potřebujeme nový vrchol, použijeme z~tohoto pole první nepoužitý. Vzhledem k~tomu, že jsme ani jedné ze struktur neimplementovali operaci \ope{Delete}, je tento postup poměrně přímočarý.

Díky naší implementaci operací \ope{Build} u~jednotlivých stromů vede tento postup k~tomu, že splay strom a červenočerný strom mají své vrcholy v~paměti v~rostoucím pořadí klíčů, tango strom a multisplay strom v~preorder pořadí jejich referenčních stromů $P$. To znamená, že je nejprve uložen kořen $P$, pak rekurzivně stejně jeho levý podstrom, a nakonec pravý podstrom.

Provedli jsme tedy experiment, kdy jsme opět naměřili čas běhu červenočerného stromu na náhodné přístupové posloupnosti. Upravili jsme ale alokaci paměti na nové vrcholy tak, že z~pole nepřidělujeme adresy novým vrcholům sekvenčně, ale v~náhodném pořadí. Tato úprava vedla k~tomu, že průměrný čas na dotek mnohem lépe kopíruje chování splay stromu, jak je vidět na obrázku \ref{obr:randomized_rb}.

\graphfigure{randomized_rb}{Průměrný čas na přístup -- náhodná přístupová posloupnost.}

\subsubsection{Teoretické chování červenočerných stromů}

Nyní se ale ještě na chvíli vrátíme k~výpočtům. Již víme, že popisovaný problém
skutečně v~praxi nastává. V~sekci \ref{sec:cache_and_balanced} jsme ukázali, že
perfektně vyváženého stromu, jehož počet vrcholů je mocnina dvojky mínus jedna,
se týká velmi výrazně. To má jednoduchý důvod -- takový strom je velmi
pravidelný, a tyto pravidelnosti jsou založené na mocninách dvojky, které také
figurují v~ekvivalenčních třídách aliasování cache. Tango stromu, multisplay
stromu a splay stromu se tento problém vyhne právě kvůli tomu, že jejich
struktura je velmi nepravidelná (přestože u~prvních dvou existuje dokonale
vyvážený referenční strom $P$. Možná jim v~našem měření pomohly také
nepravidelnosti ve struktuře $P$ pramenící z~toho, že žádná z~měřených
velikostí nebyla příliš blízko k~mocnině dvojky.)

Přirozená otázka pak ale je, nastávají tyto problémy stejnou měrou jako
v~dokonale vyváženém stromě? Ukážeme, že rozdíly mezi červenočerným a dokonale
vyváženým stromem nám pomůžou zcela zanedbatelně -- téměř všechny vrcholy,
kterými budeme s~velkou pravděpodobností potřebovat projít, budou stále na
adresách kongruentních modulo vysoká mocnina dvojky. 

Připomeneme, že
pravděpodobnost, že bude nutné projít daným vrcholem při vyhledání rovnoměrně náhodně vybraného klíče z~tohoto stromu, je rovná velikosti
podstromu tohoto vrcholu dělené velikostí celého stromu. Dále vzhledem k~tomu,
že uvažujeme sekvenční uložení vrcholů, nebudeme pracovat pro zjednodušení
práce s~adresami vrcholů, ale s~klíči v~nich -- snadno nahlédneme, že klíče
dvou vrcholů jsou kongruentní modulo $2^h$, právě když jejich adresy jsou
kongruentní modulo $2^{h+s}$, kde $2^s$ je nejvyšší mocnina dvojky, kterou je
dělitelný počet bytů paměti, které zabírá jeden vrchol. Pak dokážeme
následující tvrzení:

\begin{tvrz}
Mějme červenočerný strom $T$, který vznikl tak, že byly jeho klíče vloženy do na počátku prázdného stromu v~rostoucím pořadí. Potom pro každé $h\in \mathbb N$ jsou vrcholy, jejichž podstrom obsahuje alespoň $2^h -1$ vrcholů, až na konstantně mnoho výjimek právě ty vrcholy, jejichž klíče jsou kongruentní s~klíčem v~kořeni modulo $2^{h-1}$.
\end{tvrz}

\begin{dukaz}
Budeme předpokládat, že $T$ má alespoň $2^h-1$ vrcholů.

Nejprve začneme s~důkazem, který jsme slíbili již na začátku této sekce.

\begin{lemma}
Mějme červenočerný strom $T$, který vznikl tak, že byly jeho klíče v~rostoucím pořadí vloženy do na počátku prázdného stromu. Pak všechny červené vrcholy z~$T$ leží buď na pravé páteři $T$, nebo jsou synové vrcholu, který leží na pravé páteři $T$.
\end{lemma}
\begin{dukaz}
Budeme postupovat indukcí podle počtu vrcholů $T$. Je zřejmé, že pro strom o~jednom vrcholu tvrzení platí. Nyní ukážeme, že pokud do takového stromu $T$ nad klíči $[n]$ vložíme klíč $n+1$, strom bude stále mít požadovanou vlastnost.

Rozmyslíme si, jak by se mohlo stát, že nám vznikne červený vrchol mimo pravou páteř nebo její sousedství. Není možné, že bychom ho tam tak přímo vložili, protože vkládáme nové maximum a tedy na pravou páteř. Při vyvažování po operaci \ope{Insert} děláme 3 typy kroků: Přebarvení černého vrcholu na červený a obou jeho synů z~červené na černou, rotace a dvojitá rotace.

Při přebarvování se nám nemůže stát, že bychom přebarvili na červeno vrchol mimo pravou páteř, protože takto přebarvujeme vždy vrchol na cestě mezi vkládaným vrcholem a kořenem.

Při vkládání nového maxima se nám nikdy nemůže stát, že bychom potřebovali provést dvojitou rotaci, protože tu provádíme pouze v~situaci, kdy při cestě z~vkládaného vrcholu do kořene střídáme směry, ze kterých se vracíme.

Při vkládání nového maxima bychom mohli chtít provést jedině rotaci doleva.
Mějme tedy vrchol $v$, jeho pravého syna $u$ a levého syna $w$. Dále mějme levého syna $x$ vrcholu $u$.
rotovaná hrana vždy leží na cestě z~nového vrcholu do kořene, víme, že $v$ a
$u$ leží na pravé páteři. Kdyby se nám po rotaci stalo, že budeme mít červený vrchol mimo pravou páteř $T$ a její sousedství, musel by před rotací být červený alespoň jeden z~vrcholů $w$ a $x$. To ale není možné. Víme, že v~případě, že jsme přistoupili k~vyvažování, musí být vrchol $v$ černý a vrchol $u$ i jeho pravý syn červený. To znamená, že vrchol $u$ musel být červený již před aktuální operací \ope{Insert}. Kdyby tedy byl vrchol $x$ červený, porušoval by již před aktuální operací \ope{Insert} spolu s~vrcholem $u$ invariant červenočerného stromu. Kdyby byl červený vrchol $w$, nevyvažovali bychom strom rotací, ale přebarvením $u$ a $w$ na černo a $v$ na červeno. 
\end{dukaz}

Z~invariantu červenočerného stromu vyplývá, že pokud v~červenočerném stromu nejsou krom možná kořene žádné červené vrcholy, musí být všechny cesty z~kořene do externího vrcholu stejně dlouhé, tedy tento strom musí být dokonale vyvážený a počet vrcholů v~něm musí být bez jednoho mocnina dvojky. To samé platí i o~podstromech jednotlivých vrcholů v~červenočerném stromu.

Před další částí důkazu nahlédneme, že v~$T$ nikdy nemůže být pravý ani levý podstrom vrcholu menší, než pravý ani levý podstrom jeho syna, krom případu, kdy nějaké z~těchto podstromů obsahují část pravé páteře $T$.

Mějme dva vrcholy $u$, $v$ stromu $T$ takové, že $u$ je bez újmy na obecnosti v~pravém
podstromu $v$. Velikost levého podstromu $u$ si označíme jako $2^{h'}-1$. Rozmyslíme si, kde leží vrcholy, jejichž klíče leží mezi klíčem
$u$ a $v$. Jedná se jednak o~celý levý podstrom $v$, jednak o~některé vrcholy
na cestě mezi $u$ a $v$, a jednak o~levé podstromy těchto vrcholů. Tyto vrcholy
sice potenciálně mohou ležet i na pravé páteři, ale kořeny všech zmíněných podstromů nikoli -- buď se jedná o~levé podstromy, nebo se vše odehrává v~levém podstromu $v$. To znamená, že velikost každého z~těchto podstromů je alespoň $2^{h'} - 1$, a tedy pokud k~ní připočteme ještě samotný vrchol, na kterém tento podstrom visí, bude dělitelná $2^{h'}$. To tedy znamená, že rozdíl mezi hodnotami v~$u$ a $v$ je dělitelný $2^{h'}$.

\begin{pozorovani}\label{poz:1}
Tím jsme ukázali, že pro všechny vrcholy platí, že rozdíl jejich hodnoty a hodnoty kořene je dělitelný hodnotou jedna plus velikost pravého či levého podstromu daného vrcholu (v~závislosti na tom, zda je daný vrchol v~levém či pravém podstromu kořene).
\end{pozorovani}

Nyní najdeme vrchol $v$, což bude vrchol s~nejvyšším klíčem takový, že jeho levý
podstrom má velikost alespoň $2^{h-1} -1$, nebo kořen, pokud žádný takový vrchol neexistuje. Rozmyslíme si, že $v$ leží na pravé páteři $T$. Strom si
rozdělíme na dvě části: $T_2$, což bude pravý podstrom $v$, a $T_1$ což bude zbytek $T$ včetně $v$.

Podmínky ze znění tvrzení si označíme jako $p_1$ a $p_2$. Vrchol tedy splňuje podmínku $p_1$, pokud jeho podstrom obsahuje alespoň $2^h-1$ vrcholů, a podmínku $p_2$, pokud je jeho hodnota kongruentní s~hodnotou v~kořeni modulo $2^{h-1}$.

Nyní ukážeme, že v~$T_1$ splňují obě podmínky přesně tytéž vrcholy, a v~$T_2$ splňuje každou z~podmínek nejvýše konstantně mnoho vrcholů.

Pro $T_1$ tedy potřebujeme dokázat dvě implikace. První z~nich, tedy že vrchol
splňující $p_1$ splňuje i $p_2$, plyne z~pozorování \ref{poz:1}. Naopak pokud
máme vrchol $u$, který nesplňuje $p_1$, najdeme jeho nejbližšího předka $w$,
který $p_1$ splňuje. Rozmyslíme si, že $w$ nemůže ležet na pravé páteři. Tento vrchol má podstrom velikosti právě $2^{h}-1$. Rozdíl
jeho hodnoty a hodnoty libovolného vrcholu v~jeho podstromě je tedy maximálně
$2^{h-1}-1$. Protože dále už víme, že $w$ splňuje i $p_2$, $u$ splňovat $p_2$
nemůže.

Na poslední část důkazu potřebujeme ještě jedno lemma.

\begin{lemma}
Každý vrchol $T$, který má oba syny, má pravý strom nejvýše pětkrát větší než levý.
\end{lemma}
\begin{dukaz}
Pro vrcholy $T$ mimo pravou páteř platí lemma triviálně. Mějme tedy vrchol $v$ na pravé páteři stromu $T$. Nechť podstrom $v$ má černou výšku $h$. Dále popíšeme případ, kdy $v$ je černý -- pro červené $v$ by popis vypadal podobně. 

Víme, že levý podstrom $v$ je buď čistě černý strom hloubky $h-1$, nebo černý strom s~červeným vrcholem v~kořeni hloubky $h$. Pro potřebu horního odhadu na poměr velikosti podstromů tedy zvolíme první variantu -- levý podstrom $v$ tedy bude obsahovat $2^{h-1} -1$ vrcholů. 

Pravý podstrom $v$ chceme naopak co největší. Pravý syn $v_1$ vrcholu $v$ tedy musí být červený. Proto musí být levý podstrom $v_1$ čistě černý a tedy obsahuje také $2^{h-1} -1$ vrcholů. Pravý syn $v_2$ vrcholu $v_1$ musí být černý, jeho levý podstrom tedy může mít červený kořen, proto může mít také $2^{h-1} -1$ vrcholů. Pravý syn $v_3$ vrcholu $v_2$ bude červený, jeho levý podstrom tedy čistě černý a tedy bude mít $2^{h-2}-1$ vrcholů. Takto bychom mohli pokračovat dál. Dojdeme k~tomu, že $v_i$ má levý podstrom velikosti $2^{h-\lceil i/2\rceil} -1$. První nepravidelnost nastane až u~$v_{2h}$. $v_{2h-2}$ je černý vrchol, jehož levý i pravý podstrom už je tvořen jen jediným červeným vrcholem, a $v_h$ tedy neexistuje.

To znamená, že počet vrcholů v~pravém podstromu $v$ je $$\sum_{i=1}^{h-1}2^{h-\left\lceil \frac i2\right\rceil} -1 +1 = 2^{h+1}-3 = 4\cdot (2^{h-1}-1) +1 \leq 5\cdot(2^{h-1}-1).$$
\end{dukaz}

Víme, že levý podstrom $v$ má velikost nejvýše $2^h-1$. Velikost $T_2$ je tedy nejvýše $5\cdot 2^h - 5$. To znamená, že v~$T_2$ může být nejvýše 9 vrcholů splňujících $p_2$. Naopak díky volbě $v$ víme, že všechny vrcholy v~$T_2$ splňující $p_1$ musí ležet na pravé páteři. Protože ale víme, že podstrom kořene $T_2$ je velký nejvýše $5\cdot 2^h-5$ a při přechodu z~otce na pravého syna se vždy podstrom aktuálního vrcholu alespoň o~šestinu zmenší, můžeme udělat nejvýše konstantně mnoho kroků, než bude podstrom aktuálního vrcholu menší než $2^h-1$. 
\end{dukaz}

Tím jsme tedy ukázali, že náš červenočerný strom trpí problémy s~asociativitou cache téměř stejnou měrou, jako by byl dokonale vyvážený.


\section{Sekvenční přístupová posloupnost}\label{sec:sequential_access_sequence}

V~této třídě zvolíme velikosti stromů i délky posloupností stejně jako v~třídě
náhodných přístupových posloupností, ale místo náhodných dat budeme stále kolem
dokola přistupovat postupně ke všem vrcholům ve stromu v~rostoucím pořadí.

Na základě teorie bychom čekali, že počet dotyků na operaci splay a multisplay
stromu bude v~$n$ konstantní. Pro tango strom očekáváme $\Theta(\oldlog\oldlog
n)$ doteků na operaci. Vzhledem k~tomu, že pro červenočerný strom byla tato
posloupnost implementována (stejně jako u~ostatních stromů) jako posloupnost
individuálních operací \ope{Find}, očekáváme stejné chování jako u~náhodné
přístupové posloupnosti, přestože v~libovolném statickém stromě obecně lze
implementovat průchod v~rostoucím pořadí v~čase $\Theta(n)$.

\graphfigure{touch_s}{Doteky vrcholů -- sekvenční přístupová posloupnost.}

Tyto předpoklady víceméně naplňuje graf na obrázku \ref{obr:touch_s}. Vidíme,
že splay strom se ustálil kolem $5.42$ dotyků na operaci, multisplay strom
kolem $8.49$ s~větším rozptylem, kdežto počet dotyků tango stromu a
červenočerného stromu napříč celým grafem stále roste. Počínaje $n\cong 1000$
se červenočerný strom dotkne nejvíce vrcholů ze všech (přesnou hodnotu nelze
určit kvůli velkému rozptylu tango stromu). Nejlepšího výsledku dosahuje pro
stromy o~26 prvcích a menší červenočerný strom, počínaje $n=37$ se nejméně
vrcholů dotkne splay strom. I~počet dotyků multisplay stromu má určitý rozptyl,
proto nelze přesně určit, pro jakou hodnotu $n$ začne mít tango strom méně
dotyků na operaci než červenočerný strom, nicméně pro $n$ mezi $1000$ a $2500$
vychází počet dotyků oběma stromům podobně, pro vyšší hodnoty má tango strom
méně dotyků.

\graphfigure{time_s}{Průměrný čas na operaci -- sekvenční přístupová posloupnost.}

Budeme-li se však koukat na čas běhu, který je zachycený v~obrázku
\ref{obr:time_s}, zjistíme, že pro $n<100$ je v~praxi nejrychlejší červenočerný
strom. To jednoduše zdůvodníme tím, že červenočerný strom při každém přístupu
každý dotčený vrchol skutečně bez modifikací navštíví. Splay strom oproti tomu
navštívenými vrcholy prochází dvakrát (jednou při hledání, jednou při cestě
zpět ke kořeni), přičemž při cestě zpět ke kořeni navíc navštívené vrcholy
modifikuje.  Pro vyšší hodnoty $n$ je ale konzistentních
$40\,\operatorname{ns}$ na operaci, kterých dosahuje splay strom, výhodnější.


Multisplay strom může jedním vrcholem projít i vícekrát a dokonce
ho i vícekrát modifikovat (první modifikaci vrcholu provede při změnách
preferovaných cest tak aby byl hledaný vrchol v~pomocném stromě s~kořenem,
druhou poté při závěrečné změně preferovaného syna hledaného vrcholu). Proto
také potřebuje přibližně $250\,\operatorname{ns}$ na operaci a rychlejší než
červenočerný strom je až pro $n>1\cdot 10^7$. Tango strom má v~jistém smyslu
současně všechny nevýhody červenočerného stromu a multisplay stromu -- stejně
jako červenočerný strom neslibuje asymptoticky konstantní čas na operaci (byť
jeho asymptotická složitost je lepší než složitost červenočerného stromu),
stejně jako multisplay strom může vrcholy při přepojování preferovaných cest i
opakovaně modifikovat. Proto není divu, že i pro největší měřená data je
dvojnásobně až trojnásobně pomalejší než červenočerný strom.

Nyní se konečně dostáváme k~nejpřekvapivějšímu z~naměřených jevů. Musíme
vysvětlit, proč to vypadá, že má tango strom a v~menší míře multisplay strom
výrazný rozptyl, a proč se tento rozptyl neprojevuje pro $n\cong 1\cdot 10^4$ a
$n\cong 5\cdot10^6$. Pro osvětlení této otázky jsme provedli ještě jedno
měření. Opět jsme měřili počet dotyků na operaci pro sekvenční přístupovou
posloupnost, ale zvolili jsme $m=2n$, a navíc jsem doteky počítali pouze během
druhého průchodu. Na druhou stranu jsme měřili všechna možná $n$ z~intervalu
$[10, 1\cdot10^4]$. Výsledky měření jsou vidět na obrázku
\ref{obr:one_by_one_seq}.

\graphfigure{one_by_one_seq}{Doteky vrcholů -- sekvenční přístupová posloupnost.}

Lokální maxima křivky tango i multisplay stromu jsou v~hodnotách ve tvaru $2^k \pm \o(1)$ pro nějaké $k$, minima v~$3\cdot 2^k\pm \o(1)$ pro nějaké $k$. 

Abychom toto chování vysvětlili, musíme si rozmyslet, jak přesně probíhá
sekvenční přístup v~tango stromu. Na konci prvního sekvenčního průchodu mají
všechny vrcholy nastavený preferovaný směr doprava. My musíme během průchodu
nastavit každému vrcholu preferovaný směr nejprve doleva, a poté zase zpět
doprava. Tedy každému vrcholu jeho preferovaný směr změníme celkem dvakrát.

Je ale důležité si uvědomit, že v~tango stromu se změny preferovaných směrů
týkají pouze vrcholů, které mají v~referenčním stromě $P$ oba syny. My v~naší
implementaci však stavíme $P$ tak, aby se pro každý jeho vrchol $v$ lišila
velikost pravého a levého podstromu vrcholu $v$ nejvýše o~jedna. To vede k~tomu, že pokud se
$n$ rovná $3\cdot2^k-1$ pro nějaké $k$, $P$ vypadá tak, že na jeho předposlední
hladině mají všechny vrcholy právě jednoho syna, a tedy v~nich nelze změny
preferovaných směrů provádět. Proto pro každé $k$ platí, že při sekvenčním
průchodu stromem libovolné velikosti mezi $2^k-1$ a $3\cdot 2^{k-1}-1$ je vždy
potřeba udělat přesně ten samý počet změn preferovaných směrů. Těchto změn je přesně
dvakrát tolik, kolik má strom vrcholů s~oběma syny, tedy $2\cdot(2^{k-1} -1)$).

Změny preferovaných směrů se ale pro vyšší hodnoty $n$ při výpočtu
průměrného počtu dotyků či průměrného času rozpočítají mezi více operací,
průměr (ať už času nebo počtu dotyků) na operaci tedy bude nižší. Pro $n$
 mezi $3\cdot 2^{k-1}-1$ a $2^{k+1}-1$ naopak přidání každého dalšího listu $P$ způsobí potřebu dvou nových změn preferovaných směrů (v~rodiči tohoto listu v~$P$) při sekvenčním průchodu tango stromem.

U~multisplay stromu to samé zcela neplatí -- změnu preferovaného směru
provádíme i pro vrcholy, které mají jednoho syna, jsou-li cílem hledání
(preferovaný syn pak může být externí vrchol). Přesto ale takové změny
preferovaných směrů způsobí doteky méně vrcholů, než změny preferovaných směrů vrcholů s~oběma syny.

Zmenšení rozptylu křivek na obrázcích \ref{obr:touch_s} a \ref{obr:time_s}
kolem $n=1\cdot 10^4$ a $n=5\cdot 10^6$ vysvětlíme snadno -- v~relevantních
částech křivky jsme se naším samplováním trefili někam mezi lokální maxima a
minima. Vzhledem k~tomu, že jak lokální minima a maxima, tak naše vzorky se
exponenciálně vzdalují, ale jedná se o~exponenciály o~různých základech, není
toto chování překvapivé.

Nyní se vrátíme zpátky k~chování tango stromu při vykonávání náhodné přístupové
posloupnosti.

Pro jednoduchost se budeme zajímat pouze o~dva extrémní případy --
případy, kdy $n$ bude buď $2^k-1$, nebo $3\cdot 2^{k-1} - 1$ pro nějaké $k$. V~obou
případech platí, že každý vrchol, který má v~$P$ oba syny, má také přesně
stejný počet vrcholů ve svém pravém a levém podstromu v~$P$. Proto pro každý
přístup (bez ohledu na stav stromu před tímto přístupem) platí, že v~každém
vrcholu $P$, kterým hledání projde, ale neskončí v~něm (krom vrcholů s~jediným
synem), nastane změna preferovaného směru s~pravděpodobností $1/2$.

V~případě, že se tedy $n$ rovná $2^k-1$ pro nějaké $k$, bude bude střední
hodnota počtu změn preferovaného směru na přístup přesně rovna jedné polovině
průměrné hloubky vrcholu v~$P$ (kde kořen má hloubku 0). Pokud se $n$ naopak rovná
$3\cdot2^{k-1}-1$ pro nějaké $k$, bude tato hodnota sice také rovna průměrné
hloubce vrcholu, ale s~tím, že listům z~poslední hladiny $P$ počítáme hloubku o~jedna menší.

Všimneme si, že na rozdíl od sekvenční přístupové posloupnosti je při vykonávání náhodné
posloupnosti střední hodnota počtů změn preferovaných směrů na přístup v~$n$
monotónně rostoucí. To je nepřekvapivé -- i pro $n$ mezi $2^k-1$ a $3\cdot
2^{k-1}-1$ stále přidáváme nové vrcholy do spodní vrstvy $P$ a tedy zvyšujeme
pravděpodobnost, že daný dotaz povede hlouběji do stromu $P$, a to i přesto, že
hloubku nejhlubších listů počítáme jako o~jedna menší.

\graphfigure{variance_tango_random}{Vliv vzdálenosti od mocniny dvou na počet dotyků vrcholu na přístup -- náhodná přístupová posloupnost.}

Na závěr ještě ukážeme, jak výrazné rozdíly v~praxi tento jev na počtu dotyků
způsobí. Na grafu na obrázku \ref{obr:touch_r} je jev velmi nezřetelný, to je ale
částečně tím, že křivka tango stromu na tomto grafu sama o~sobě poměrně prudce
roste, v~čemž se může leccos ztratit. Proto jsme tento jev alespoň částečně
izolovali. Jednotlivé naměřené hodnoty pro tango strom v~grafu na obrázku
\ref{obr:touch_r} označíme jako $h_1, h_2, \dots, h_{50}$. Potom v~grafu
\ref{obr:variance_tango_random} zobrazíme hodnoty $$h'_i =
\frac{\frac{h_{i-1}+h{i+1}}2-h_i}2.$$ Jinými slovy, podíváme se, jak daleko je
$h_i$ od aritmetického průměru jeho sousedů. Protože pak ale odečítáme lokální
maximum od průměru dvou lokálních minim nebo naopak, výslednou hodnotu ještě
vydělíme dvěma. Tím dostaneme určitou představu o~tom, jak daleko je daný
vzorek od v~nějakém smyslu vyhlazené křivky. V~grafu vidíme maxima kolem $1/2$,
což zhruba odpovídá očekávání. 

\section{Bit reversal posloupnost na páteři}

V~této sekci prozkoumáme posloupnost, která nad danou množinou klíčů $[n]$
vznikla tak, že jsme vystavěli dokonale vyvážený strom nad $[n]$, vzali jeho
pravou páteř a nad ní aplikovali myšlenku bit reversal posloupnosti. Takto
vzniklou posloupnost potom zřetězíme samu se sebou tolikrát, kolikrát je nutné
k~dosažení cílové délky. Vzhledem k~tomu, že délka jednotlivých iterací
posloupnosti se mění až s~dvojnásobným zvětšením $n$, zvolili jsme jako
testované hodnoty $n$ právě mocniny dvojky od $2^3$ po $2^{26}$.

Jednotlivé iterace bit reversal posloupnosti jsou velmi krátké, proto jsme se
rozhodli pro konstantní hodnotu $m=1\cdot10^7$. Přímočaré měření takovýmto
způsobem by ale vedlo k~neférovému znevýhodnění splay stromu -- ostatní stromy
mají již po vystavění logaritmickou hloubku, splay strom má ale po sekvenčním
vložení prvků hloubku lineární. Pro nejvyšší hodnotu $n$ se tedy první přístup
dotkne $2^{26} \cong 6\cdot10^7$ vrcholů, což je dost na to, aby se nevhodný
výchozí tvar stromu projevil v~celém měření. Proto jsme se u~splay stromu
rozhodli do měření nezapočítávat prvních 100 operací. Těchto 100 operací
s~rezervou stačí k~tomu, abychom operacemi \ope{Splay} přesunuli všechny prvky,
které se v~posloupnosti nachází, do okolí kořene.

\graphfigure{touch_i}{Doteky vrcholů -- Bit reversal posloupnost na páteři.}
\graphfigure{time_i}{Průměrný čas na přístup -- Bit reversal posloupnost na páteři.}

Při vykonávání této posloupnosti se nejlépe chová tango strom. To je vidět i na
obrázcích \ref{obr:touch_i} a \ref{obr:time_i}. To není příliš překvapivé -- všechny
navštěvované vrcholy jsou na jedné cestě z~kořene referenčního stromu $P$ do
listu. Proto při několika prvních přístupech pospojujeme všechny navštěvované
vrcholy do jediného logaritmicky velkého červenočerného stromu a další přístupy
již vykonáváme bez modifikací stromu. Na druhou stranu díky Interleave Bound
víme, že tuto posloupnost rychleji, než v~čase $\Theta(\oldlog \oldlog n)$
vykonat nelze -- není tedy šance, že by byl tango strom asymptoticky překonán
například splay stromem. 

Splay strom se chová jen o~něco hůře -- rozdíl je podobný, jaký jsme pozorovali
u~náhodné přístupové posloupnosti mezi červenočerným stromem a splay stromem.
To je očekávané. Oba stromy totiž pracují (krom několika úvodních přístupů)
pouze s~vrcholy, jejichž hodnoty patří do přístupové posloupnosti, tango strom
je ale nijak nemodifikuje. 

V~červenočerném stromu jsou vrcholy, k~nimž přistupujeme, rozmístěny víceméně
náhodně -- jedná se o~vrcholy, jejichž hodnota je menší nebo rovna polovině
$n$, což vede k~tomu, že budou umístěny spíše v~levé, lépe vyvážené části
stromu. Nic silnějšího o~nich ale červenočerný strom nezaručuje. Proto může být
trochu překvapivé, že se v~našem měření červenočerný strom choval lépe než
multisplay strom. Multisplay strom má totiž lepší asymptotickou složitost.
Bohužel ale multisplay stromy kromě toho, že fungují  podobně jako splay
stromy, ještě provádějí změnu preferovaného syna v~cílovém vrcholu. To znamená,
že při vykonávání této posloupnosti vykonají v~průměru necelé dvě změny
preferovaného směru na přístup -- pokud cílový vrchol není list referenčního
stromu $P$, musíme v~něm změnit preferovaný směr. Všechny změny preferovaných
směrů jsou ale ve skutečnosti nežádoucí a bude je při dalších přístupech nutné
vrátit.

\def\doublegraphfigure#1#2#3{
\begin{figure}[h!]
\centering
\includegraphics[width=.8\hsize]{graphs/#1_touch_#2}
\includegraphics[width=.8\hsize]{graphs/#1_time_#2}
\caption{#3}
\label{obr:#1_#2}
\end{figure}
}

\section{Podmnožinové posloupnosti}

Nyní se podíváme na dva typy podmnožinových posloupností, náhodnou a sekvenční
podmnožinovou přístupovou posloupnost. Oba typy posloupností vypadají tak, že
vždy nejprve zkonstruujeme multimnožinu $M$ velikosti $k$ tak, že $k$-krát
zvolíme rovnoměrně náhodný prvek $[n]$ a přidáme ho do naší multimnožiny. Prvky
přístupové posloupnosti pak vybíráme pouze z~multimnožiny $M$. 

Při tvorbě
náhodné podmnožinové posloupnosti vybíráme prvky rovnoměrně náhodně, tedy tak, že pro každé
$i\leq m$ a $j\leq k$ je pravděpodobnost, že $i$-tý prvek přístupové
posloupnosti bude $j$-tý prvek $M$ rovna počtu výskytů tohoto prvku v~$M$
dělenému $k$.

Do sekvenční podmnožinové posloupnosti vybíráme prvky v~(cyklickém) rostoucím pořadí tak, že vícenásobný výskyt prvku v~$M$ vede k~vícenásobným po sobě jdoucím výskytům daného prvku ve výsledné posloupnosti. 

Pro měření jsme zvolili opět podobně jako u~sekvenční a náhodné přístupové
posloupnosti hodnoty $n$ tak, aby se dvě po sobě jdoucí velikosti stromu od
sebe lišily v~poměru $1:\sqrt[7]{10}$, ale jako maximální hodnotu $n$ jsme
zvolili $1\cdot10^7$. Nejmenší měřená hodnota $n$ byla $10$. Hodnotu $m$ jsme
opět zvolili jako $\max(1\cdot10^7, 10\cdot n)$. K~měření jsme zvolili hodnoty
$k$ rovny kladným celočíselným mocninám deseti takovým, aby $k<n$.

Vzhledem k~tomu, že pro každou strukturu a každý typ posloupnosti máme šest
různých hodnot $k$, rozhodli jsme se pro přehlednost vytvořit grafy pro každou
strukturu a typ posloupnosti zvlášť. Do každého z~grafů jsme také přidali
křivku příslušné \uv{nepodmnožinové} posloupnosti s~popiskem $k=n$, přestože se
nejedná o~zcela přesný popis -- tyto posloupnosti jsou speciálním případem
podmnožinových posloupností, kdy $M$ není multimnožina ale množina a každý
prvek $[n]$ v~ní tedy musí být právě jednou.

\subsection{Červenočerný strom}

\doublegraphfigure{rb}u{Červenočerný strom -- náhodná podmnožinová posloupnost}
\doublegraphfigure{rb}b{Červenočerný strom -- sekvenční podmnožinová posloupnost}

Červenočerný strom se co se týče počtu doteků chová stejně jako u~všech
ostatních posloupností -- všechny přístupové posloupnosti potřebují $\log n \pm
\o(1)$ doteků na operaci. Jediný detail, kterého bychom si mohli všimnout, je,
že pro nejmenší $k=10$ je ve tvaru křivky mnoho nepravidelností -- pro takto
malé $k$ je totiž nejvyšší pravděpodobnost, že průměrná hloubka $k$ náhodně
vybraných vrcholů bude daleko od střední hodnoty této náhodné proměnné. 

Na grafu času na operaci vidíme, že pro $n<1\cdot10^4$ se všechny posloupnosti
chovají přibližně stejně i podle tohoto hlediska. Pro $1\cdot10^4 < n < 3\cdot
10^5$ se začíná projevovat odlišné chování k~L1 a L2 cachím, a pro $n > 3\cdot
10^5$ se projeví odlišné chování k~L3 cachi. Podle tohoto chování můžeme
všechny posloupnosti rozdělit do dvou hlavních skupin s~několika ojedinělými
případy zapadajícími někam mezi tyto skupiny -- skupina posloupností, kde čas
na operaci roste i pro vysoká $n$ stejně jako pro nízká a pro $n=10\cdot 7$
potřebujeme na operaci v~průměru mezi pětinou a čtvrtinou mikrosekundy, a
skupina, kde čas na operaci začíná kolem $n=3\cdot 10^5$ prudce růst a pro
nejvyšší testované hodnoty potřebujeme na  
operaci téměř mikrosekundu.

Připomeňme, že nad náhodnými (jak normálními, tak podmnožinovými) přístupovými posloupnostmi je nejlepší cachovací strategie (pokud se nám nevejde celá relevantní část stromu do cache) mít v~cache co nejvíc prvních hladin. Otázka ale je, jak velká je relevantní část stromu (tedy množina vrcholů, kterými musíme alespoň jednou projít) v~závislosti na $k$ a $n$. To je náhodná veličina která nezávisí pouze na $n$ a $k$, ale i na přesné volbě $M$. My si zde pro představu uděláme pouze horní odhad velikosti relevantní části stromu pro dokonale vyvážený binární strom pro případ, kdy $k\leq2\cdot n$. 

Největší možné relevantní části stromu dosáhneme, když budou všechny prvky $M$ v~listech stromu a v~nějakém smyslu pravidelně rozprostřené -- pod tím si můžeme představit, že se pro libovolný vrchol bude počet prvků $M$ v~pravém a levém podstromu tohoto vrcholu lišit nejvýše o~jedna. Potom do relevantní části stromu plně patří prvních $\log k$ hladin stromu. K~nim navíc musíme přidat cestu z~každého vrcholu na $(\log k)$-té hladině do nějakého listu. Těchto cest je $k$ a každá z~nich má délku $\log n - \log k = \log (n/k)$. Relevantní část stromu  má tedy celkem $2k + k\cdot \log (n/k)$ vrcholů.

Vzhledem k~tomu, že jsme si již všimli, že náš červenočerný strom má tvarem k~dokonale vyváženému blízko, můžeme tato pozorování na červenočerný strom uplatnit.

Nyní se musíme zamyslet, co se stane, když se relevantní část stromu nevejde do
cache. Pro náhodné podmnožinové posloupnosti platí, že v~cachi budeme mít
uložených prvních $\log k$ hladin stromu (nebo tolik, kolik se vejde). Cesty od
konce této uložené části patří každá pouze jednomu vrcholu, s~vysokou
pravděpodobností nám je tedy před dalším přístupem k~tomuto vrcholu ostatní
operace z~cache vyhází. Dává tedy smysl, že jakmile se nám relevantní část
stromu do cache nevejde, musíme při většině přístupů cestu od přinejmenším
$(\log k)$-té hladiny do cílového vrcholu načíst z~paměti. Pro vyšší nižší
hodnoty $k$ jsou tyto cesty sice delší, ale relevantní část stromu je menší.
Proto můžeme mít pro nižší hodnoty $k$ v~paměti uložených více těchto cest.
Vzhledem k~tomu, že navíc snižujeme hodnotu $k$, se podíl počtu všech všech
cest a cest, které můžeme mít uložené, zvětšuje s~klesajícím $k$. Na základě
měření na grafech \ref{obr:rb_u} můžeme říct, že efekty prodlužování cest a
větší pravděpodobnost uložení cest se zhruba vyrovnají a pro $k > 1\cdot 10^4$
se již průměrný čas na přístup téměř nemění.

Pro sekvenční přístupovou posloupnost je situace trochu jiná -- tam se LRU
cache chová tak, že pokud se ani relevantní část pravého či levého podstromu
kořene nevejde celá do cache, bude při každém sekvenčním průchodu postupně celý
strom krom kořene z~cache vyhozen a zase načten. Proto bude průměrný čas na
operaci záviset na poměru velikosti relevantní části stromu a počtu operací
v~jednom sekvenčním průchodu stromem. S~využitím odhadu výše tento poměr
odhadneme jako $2 + \log(n/k)$ vrcholů relevantní části stromu na operaci. Tedy
paradoxně by měl být strom pro vyšší hodnoty $k$ rychlejší.

Tento závěr je konzistentní s~měřením -- na obrázku \ref{obr:rb_b} vidíme, že pro $k=1\cdot10^3$ se nám
ještě celá relevantní část stromu vešla do L3 cache. Pro $k=1\cdot 10^4$ se
těsně nevešla a proto se také jedná o~nejpomalejší z~měřených posloupností. Pro
dále se zvětšující $k$ se jednotlivé operace dále zrychlují a to až k~normální
sekvenční přístupové posloupnosti, která potřebuje načíst z~paměti v~průměru
pouze jeden vrchol na přístup. 

\subsection{Splay strom}

\doublegraphfigure{splay}u{Splay strom -- náhodná podmnožinová posloupnost}
\doublegraphfigure{splay}b{Splay strom -- sekvenční podmnožinová posloupnost}

Chování splay stromu při vykonávání náhodné podmnožinové přístupové
posloupnosti nepřináší žádné překvapení -- na grafech na obrázku \ref{obr:splay_u}
vidíme, že u~splay stromu skutečně při vykonávání náhodné přístupové
posloupnosti záleží pouze na $k$ a nikoli na $n$. Na křivce času pro $k =
1\cdot 10^5$ vidíme okamžik, kdy už se celý splay strom nevešel do cache.
Vzhledem k~tomu, že množinu $M$ z~vrcholů stromu vybíráme náhodně, již od
tohoto bodu nestačí, aby se nám do cache vešlo $k$ vrcholů -- musíme počítat
s~tím, že každý vrchol zabere celý cachový řádek (protože s~ním v~jednom řádku
velmi pravděpodobně nebude žádný další vrchol z~$M$).

Na začátku každé křivky se projeví ještě efekt, který můžeme nejlépe zkoumat
pro $k = 1\cdot 10^6$. Nejmenší hodnota $n$, pro kterou zkoumáme $k=1\cdot
10^6$ je $n= 1\cdot 10^6\cdot \sqrt[7]{10} \cong 1.3\cdot10^6$. Protože je $n$
jen o~malinko větší než $k$, bude v~$M$ poměrně hodně duplicitních vrcholů.
Vzhledem k~tomu, že chování splay stromu závisí pouze na počtu unikátních
vrcholů, není překvapivé, že časy přístupů i počty doteků jsou nižší, když jsou hodnoty $k$ a $n$ blízko sebe.

Tento efekt se ještě výrazněji projeví na obrázku \ref{obr:splay_b} při vykonávání sekvenční přístupové posloupnosti -- tam totiž duplicitní prvek v~$M$ znamená, že v~přístupové posloupnosti bude to samé číslo vícekrát za sebou. Pouze první z~takových přístupů pak proběhne standardně, další se už dotknou pouze kořene.

Na grafu průměrného času na operaci při vykonávání podmnožinové sekvenční
přístupové posloupnosti vidíme ještě jednu věc, která stojí za povšimnutí --
výrazné oddělení křivek pro $k=1\cdot 10^5$ a $k=1\cdot 10^6$ od ostatních. To opět souvisí s~cachí. 

Při sekvenčním průchodu splay stromem se každého vrcholu dotkneme v~průměru o~něco víc než pětkrát, jak jsme již experimentálně zjistili v~dřívějších sekcích. Přinejmenším při některých z~těchto dotyků se dotýkáme mnoha hodnotou sousedících vrcholů po sobě -- například po poslední operaci jednoho sekvenčního průchodu jsou všechny vrcholy na levé páteři stromu a první operace z~následujícího průchodu se jich dotkne všech dvakrát (ač my toto počítáme jako jeden dotek) nejprve v~sestupném a pak ve vzestupném pořadí. 

Pokud $k < 1\cdot10^5$, vejde se nám celá relevantní část stromu do cache. Když
vykonáváme (\uv{nepodmnožinovou}) sekvenční přístupovou posloupnost se stromem,
který se nám do cache nevejde, můžeme stále využít toho, že máme vrcholy
v~paměti uložené v~sekvenčním pořadí -- pokud se dotýkáme většího počtu
sousedících vrcholů hned po sobě, nemusíme kvůli každému vrcholu načítat nový
cachový řádek. Cachové řádky mají 64 bytů, jeden vrchol splay stromu zabírá 24
bytů, tedy pokud se dotkneme mnoha sousedících vrcholů současně, stačí z~paměti
načíst $3/8$ cachového řádku na vrchol. Pokud ale provádíme podmnožinovou
sekvenční přístupovou posloupnost, vrcholy vybíráme z~naalokovaných náhodně,
tedy kvůli většině z~nich musíme načíst zvláštní cachový řádek. Navíc se může
stát, že ani nenačítáme po sobě jdoucí řádky, tedy nevyužíváme faktu, že paměť
RAM je obvykle rychlejší při sekvenčním než při náhodném průchodu. Proto
potřebuje podmnožinový sekvenční průchod splay stromem v~situaci, kdy $k$ je
větší než počet řádků cache, ale výrazně menší než $n$, na přístup výrazně víc
času než standardní sekvenční průchod. 

\subsection{Tango strom}

\doublegraphfigure{tango}u{Tango strom -- náhodná podmnožinová posloupnost}
\doublegraphfigure{tango}b{Tango strom -- sekvenční podmnožinová posloupnost}

Dále jsme zkoumali chování tango stromu při vykonávání podmnožinové náhodné
přístupové posloupnosti. Na grafu s~logaritmickou osou $x$ a lineární osou $y$
znamená z~$[0,0]$ do $[a,a]$ pro kladné $a$ logaritmický růst zobrazované
proměnné. Na prvním grafu na obrázku \ref{obr:tango_u} vidíme, že křivka standardní náhodné přístupové
posloupnosti je mírně konvexní, strom tedy potřebuje více než logaritmický
počet dotyků na operaci. Z~teorie víme, že potřebuje $\Theta(\oldlog\oldlog n
\cdot \oldlog n)$ doteků. Naopak křivky podmnožinových posloupností jsou mírně
konkávní, přesto rostoucí. Popsaná pozorování odpovídají očekávání, že v~tomto
případě bude tango strom potřebovat $\Theta(\oldlog\oldlog n \cdot \oldlog k)$
doteků na operaci. Grafy průměrného času na operaci velmi přesně kopírují grafy potřebného
množství doteků.

Nyní se podíváme na podmnožinovou sekvenční přístupovou posloupnost.
Připomeneme, že při normálním sekvenčním průchodu splay stromem musíme provést
celkem dvě změny preferovaného směru vrcholu v~každém vrcholu, který je
v~referenčním stromu $P$ vnitřní. Například pro $n=2^h-1$ pro nějaké $h$ se jedná
celkem o~$n - 1$ změn. Navíc až $(n+1)/2$ z~nich bude ve vrcholech, které jsou
v~$P$ na předposlední hladině. Takové změny preferovaných směrů mohou provést
poměrně málo doteků vrcholů, protože vrcholy $r'$ a $\ell$ pro tyto změny jsou
ve stromu blízko sebe a cesty mezi těmito vrcholy a kořenem sdílí většinu
vrcholů. 

Při vykonávání podmnožinové sekvenční přístupové posloupnosti je situace trochu
jiná. Při popisu chování červenočerného stromu jsme udělali odhad, jak nejhůře
může vypadat relevantní podstrom dokonale vyváženého stromu. Tento odhad znovu
aplikujeme na strom $P$. Při podmnožinovém sekvenčním průchodu tango stromem
musíme udělat dvě změny preferovaného směru v~každém vrcholu relevantního
podstromu stromu $P$, který má v~tomto podstromu oba syny. To může být až $k-1$
vrcholů. Navíc více než polovina z~nich může být na $(\log k)$-té hladině $P$
(a ostatní ještě výše).

Dostáváme tedy, že při standardním sekvenčním průchodu musíme udělat pouze
přibližně jednu změnu preferovaného směru na přístup, a navíc víc než polovina
těchto změn odpojuje a zase připojuje pomocné stromy o~jediném vrcholu. Při
podmnožinovém sekvenčním průchodu může být potřeba udělat až téměř dvě změny
preferovaného směru na přístup a navíc všechny tyto změny mohou připojovat
části preferovaných cest o~alespoň $\lfloor\log (n/k)\rfloor$ vrcholech. Tím
lze zdůvodnit, že podmnožinová sekvenční přístupová posloupnost vede podle obrázku \ref{obr:tango_b} oproti
standardní sekvenční přístupové k~násobně více dotykům na operaci, a tento
potřebný počet dotyků dále s~rostoucím $n$ (nebo taky pro konstantní $n$
s~klesajícím $k$). 

Připomeneme, že naše implementace tango stromu má na začátku všechny pomocné
stromy o~jediném vrcholu, tedy vrcholy na počátku nemají nastavený preferovaný
směr. To znamená, že při vykonávání podmnožinové sekvenční posloupnosti se
skutečně dotkneme právě těch vrcholů, které jsou v~relevantním podstromu stromu
$P$. Proto není překvapivé, že se při vykonávání této posloupnosti pro nízké
hodnoty $k$ vejde celá relevantní část stromu do cache (různých jejích úrovní
v~závislosti na přesné hodnotě $k$). To je vidět na nižších průměrných časech na
operaci pro $k \leq 1\cdot 10^3$, jinak ale opět průměrné časy na operaci
odpovídají průměrným počtům dotčených vrcholů.

\subsection {Multisplay strom}

\doublegraphfigure{multisplay}u{Multisplay strom -- náhodná podmnožinová posloupnost}
\doublegraphfigure{multisplay}b{Multisplay strom -- sekvenční podmnožinová posloupnost}

K~multisplay stromu už toho nelze mnoho dodat -- neprojeví se zde žádný efekt,
o~kterém bychom již nemluvili při rozebírání chování tango stromu. Multisplay
strom však dosahuje ve všech směrech o~konstantu lepších výsledků než tango
strom. Například pro $k=1\cdot10^3$ a $n=1\cdot 10^7$ potřeboval tango strom
při vykonávání podmnožinové náhodné přístupové posloupnosti na operaci zhruba
90 doteků a 5 mikrosekund. Podle obrázku \ref{obr:multisplay_u} stačilo
multisplay stromu na tutéž operaci přibližně 50 doteků a 3 mikrosekundy.
Podobně při vykonávání podmnožinové sekvenční přístupové posloupnosti
potřeboval tango strom asi 42 doteků a 2 mikrosekundy na operaci, multisplay
stromu stačilo 27 doteků a 660 nanosekund, jak je vidět na obrázku
\ref{obr:multisplay_b}.

\subsection{Porovnání}

Popsali jsme chování jednotlivých struktur na jednotlivých typech posloupnosti.
Teď ještě musíme toto chování zběžně porovnat. Již jsme řekli, že multisplay
strom se chová kvalitativně podobně jako splay strom. Splay strom i
červenočerný strom však dosahuje pro libovolnou kombinaci parametrů $n$ a $k$
lepších výsledků než multisplay strom. Pro sekvenční podmnožinovou přístupovou
posloupnost dosahuje opět pro libovolnou kombinaci parametrů kromě velmi
nízkých hodnot $n$  lepších výsledků splay strom (byť pro $n=1\cdot 10^7$ a
$k=1\cdot 10^6$ s~poměrně malým rozdílem) -- chování pro velmi nízké hodnoty
$n$ je ale pro oba stromy podobné, jako při provádění standardní sekvenční
přístupové posloupnosti a tedy ho není nutné znovu popisovat. Pro podmnožinové
náhodné přístupové posloupnosti platí, že se splay strom chová lépe než
červenočerný strom, pokud je $k$ alespoň stokrát menší než $n$. V~případě, že
se strom nevejde celý do cache, se potom červenočerný strom pro $k$
přibližující se k~$n$ zrychluje. Chování splay stromu naopak závisí pouze na
$k$ a čas potřebný na operaci s~ním logaritmicky roste.

