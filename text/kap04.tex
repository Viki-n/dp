\chapter{Výsledky}

V této kapitole popíšeme chování naší implementace stromů na několika třídách přístupových posloupností.

\section{Náhodná přístupová posloupnost}

Třída náhodných přístupových posloupností vypadá tak, že pro danou velikost stromu pomocí funkce {\tt rand} z hlavičkového souboru {\tt stdlib.h} vyrobíme náhodnou posloupnost přirozených čísel z intervalu $[0,n)$. Pro měření jsme zvolili velikosti stromů od $10$ do $1\cdot 10^8$, kde každá další testovaná velikost byla $\sqrt[7]{10}$. Takovýto krok jsme zvolili proto, že s ním v celém rozsahu měříme celkem 50 různých velikostí stromu, tedy dosáhneme rozumného kompromisu mez počtem data pointů a celkovou dobou měření. Délka posloupnosti byla vždy $\max(1\cdot10^7, 10\cdot n)$. Takové číslo jsme zvolili proto, abychom snížili pravděpodobnost, že dostaneme náhodnou posloupnost, jejíž čas běhu a počet dotyků na operaci budou daleko od příslušné střední hodnoty, abychom i pro větší $n$ alespoň jednou navštívili téměř všechny prvky, a aby byla posloupnost i pro malá $n$ dost dlouhá, aby nepřesnost měření času (která může dosahovat až řádově desítek milisekund) nehrála významnou roli. 

\def\graphfigure#1#2{
\begin{figure}[h!]
\centering
 \makebox[\textwidth][c]{\includegraphics{graphs/#1}}
\caption{#2}
\label{obr:#1}
\end{figure}
}

\graphfigure{touch_r}{Doteky vrcholů -- náhodná přístupová posloupnost.}


Při vlastním měření jsme ale neměřili dvě nejvyšší hodnoty $n$ pro tango strom kvůli vysoké časové náročnosti testů.

Na základě teorie bychom čekali, že počet doteků na operaci u multisplay, splay a červenočerného stromu poroste logaritmicky, u tango stromu navíc ještě s faktorem $\log\log$ navíc, což měření podporují, viz \ref{obr:touch_r}.
\let\oldlog\log
\def\log{\oldlog_2}

Na grafu vidíme, že červenočerný strom potřebuje konzistentně o zhruba
půl doteku na operaci méně než $\log n$, které jsme do grafu přidali pro
porovnání. Splay strom potom potřebuje zhruba $1.28\cdot \log n$ doteku,
multisplay strom dokonce $3.35\cdot \log n$. U tango stromu jsme na našem vzorku naměřili v průměru $4.88\cdot \log n$ doteků, což ale není příliš zajímavé -- u ostatních stromů očekáváme, že bude počet doteků na operaci dělený logaritmem konvergovat k nějaké konstantě, u tango stromu toto očekávání nemáme. Nejvyšší zaznamenaný počet dotyků pro tango strom byl v testu pro $n \cong 5.18\cdot 10^7$. V tomto testu jsme naměřili asi $6.81\cdot \log n$ doteků na operaci. Pro srovnání, $\log\log 5.18\cdot 10^7 \cong 4.68$. 

Dále si všimneme mírného rozvlnění křivky tango stromu kolem $n=5\cdot 10^3$ a znovu kolem $n=1\cdot 10^6$. Tento efekt se ale výrazněji projeví při sekvenčních přstupech, které budeme diskutovat v následující sekci, s jeho vysvětlením tedy počkáme tam. 

