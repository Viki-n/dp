%%% Fiktivní kapitola s ukázkami citací

\chapter{Cíle práce}

Cílem práce je určit, která struktura binárního vyhledávacího stromu je v praxi nejefektivnější, kde efektivitu budeme měřit jednak počtem navštívených vrcholů, jednak časem běhu při vykonávání několika konkrétních posloupností. Zkoumat budeme červenočerný strom, splay strom, tango strom a multisplay strom. Budeme postupovat následovně:

\begin{enumerate}
\item Vytvoříme implementaci řečených datových struktur v jazyce C.
\item Navrhneme několik sad syntetických přístupových posloupností pro různě velké datové struktury.
\item Změříme chování implementací datových struktur na sadách posloupností. Poté provedeme diskusi.
\item Nalezneme program, který obsahuje implementaci nějakého binárního vyhledávacího stromu. S pomocí tohoto programu vyrobíme přístupovou posloupnost z reálného světa. S touto posloupností budeme opakovat předchozí bod.
\end{enumerate}
