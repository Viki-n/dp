%%% Šablona pro jednoduchý soubor formátu PDF/A, jako treba samostatný abstrakt práce.

\documentclass[12pt]{report}

\usepackage[a4paper, hmargin=1in, vmargin=1in]{geometry}
\usepackage[a-2u]{pdfx}
\usepackage[czech]{babel}
\usepackage[utf8]{inputenc}
\usepackage[T1]{fontenc}
\usepackage{lmodern}
\usepackage{textcomp}

\begin{document}

%% Nezapomeňte upravit abstrakt.xmpdata.
V této práci jsme srovnali několik variant binárního vyhledávacího stromu, které se
blíží dynamické optimalitě: tango stromy,
multisplay stromy a splay stromy. 
Experimentálně jsme prozkoumali chování těchto tří typů stromů a červenočerných stromů. Měřili jsme počet navštívených vrcholů na
operaci a také čas běhu na reálném hardwaru. Ukázali jsme, že tango strom a multisplay strom jsou ve většině případů méně efektivní než červenočerný a
splay strom. V~chování splay stromu a červenočerného stromu hrály nečekaně velkou roli efekty cache.


\end{document}
