\chapter*{Závěr}
\addcontentsline{toc}{chapter}{Závěr}

V~práci jsme představili teoretický model dynamického binárního vyhledávacího stromu a několik starších výsledků na cestě k~dynamické optimalitě.

Vytvořili jsme vlastní implementaci čtyř představených stromů, červenočerného, splay, tango a multisplay. Poté jsme navrhli několik tříd přístupových posloupností, na kterých jsme zkoumali, kolik doteků vrcholů a kolik času který ze zmíněných stromů potřebuje na vykonání jedné operace \ope{Find}. 

Zjistili jsme, že nad náhodnými přístupovými posloupnostmi se nejlépe chová
červenočerný strom. Objevili jsme však překvapivý potenciální problém
červenočerného stromu -- zjistili jsme, že pokud jeho vrcholy naalokujeme a
vložíme do stromu v~sekvenčním pořadí, narazíme na problémy související
s~asociativitou cache. Tyto problémy se začnou projevovat ve chvíli, kdy je kapacita cache v bytech menší než počet vrcholů stromu
vynásobený nejvyšší mocninou dvojky, která dělí množství bytů, které v~paměti
zabírá jeden vrchol.

Nad sekvenčními a sekvenčními podmožinovými přístupovými posloupnostmi se
naopak nejlépe choval splay strom. U~něj jsme naopak narazili na zpomalení,
když vrcholy v~rostoucím pořadí uloženy nebyly Toto zpomalení však nebylo dost
velké na to, aby byl splay strom červenočerným stromem překonán. Pro náhodné
podmnožinové posloupnosti byl splay strom srovnatelně rychlý jako červenočerný
strom, pokud byla velikost podmnožiny $M$, z níž byly cílové vrcholy přístupů vybírány, zhruba stokrát menší, než kolik měl strom vrcholů. Pro nižší hodnoty $|M|$ byl rychlejší splay strom,
pro vyšší červenočerný strom. 

Pro vykonávání proměnlivě podmnožinových přístupových posloupností se ukázal
obecně vhodnější červenočerný strom. Splay strom byl rychlejší pouze pro volbu
parametru $|M|$ takovou, aby se $|M|$ vrcholů sice vešlo do cache, ale pouze těsně.

Při vykonávání všech zmíněných posloupností byl tango strom časově s~velkým
rozdílem nejhorší, i když počtem dotčených vrcholů při vykonávání sekvenční
přístupové posloupnosti se zařadil před červenočerný strom. Multisplay strom se
choval výrazně lépe než tango strom, ale stále hůře než oba zbývající stromy.

Našli jsme také jednu velmi specifickou přístupovou posloupnost, při jejímž vykonávání se tango strom choval nejlépe ze všech stromů. Najít přístupovou posloupnost, na které by se choval nejlépe multisplay strom, se nám nepovedlo. U~několika různých tříd posloupností jsme nabyli dojmu, že multisplay stromu škodí, že na konci každé operace \ope{Find} provede ještě změnu preferovaného směru v~cílovém vrcholu -- mohlo by být zajímavé zopakovat experimenty s~variantou multisplay stromu, který tuto změnu neprovádí. Protože ale na této změně závisí důkazy asymptotického chování struktury, nemuseli bychom ji vynechávat, pouze nahradit operací \ope{Splay} zavolanou na cílový vrchol -- pak by vlastnosti struktury tak, jak je  představili \citet{multisplay}, zůstaly zachovány.

Dále by bylo zajímavé najít přístupovou posloupnost z reálné aplikace binárních vyhledávacích stromů a otestovat stromy na ní, jak jsme ale již zmínili, není jednoduché vhodnou posloupnost najít.

