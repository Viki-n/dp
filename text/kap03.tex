\chapter{Metodika}

V této kapitole popíšeme některé detaily implementací binárních vyhledávacích stromů, které vznikly pro potřeby této práce (ne však rozhraní těchto implementací -- to popíšeme později v příloze práce). Dále popíšeme metodiku měření času a počtu dotčených vrcholů.

\section{Implementace}

Všechny testované struktury jsme implementovali v jazyce C. Ten byl zvolen proto, že umožňuje kontrolovat, co přesně měříme -- měření nám nepokazí ani interní fungování interpreteru jako by hrozilo, kdybychom zvolili například Python, ani zásahy garbage collectoru, které by se nám nevyhnuly například v Javě.

Při implementaci testovaných struktur jsme pro každou strukturu naimplementovali operaci \ope{Find}. Dále jsme pro červenočerný strom a splay strom napsali operaci \ope{Insert}, pro zbylé dva stromy potom operaci \ope{Build}. Protože ale stavbu struktury nechceme měřit, tak před měřením pro červenočerný a splay strom struktury nejprve vybudujeme tím, že do nich postupně vložíme všechny klíče, které mají obsahovat, v rostoucím pořadí. 

Dále jsem se odchýlili od modelu -- ve snaze minimalizovat množství paměti zabírané jednotlivými vrcholy jsme se rozhodli nemít ve vrcholech ukazatele na rodiče a místo toho si ukazatele na rodiče všech vrcholů mezi aktuálním vrcholem a kořenem držet na zásobníku. Čistě pro zjednodušení algoritmu jsem se také rozhodli operace \ope{Split} a \ope{Join} červenočerného stromu, které jsou interně využívány operací \ope{Find} tango stromu, neimplementovat pomocí rotací hran, ale přímo tak jak byly popsány v sekci o červenočerném stromu. 

\citet{tango} ve svém článku o tango stromu také mluví o tom, že jako součást přístupu vždy změní preferovaný směr v cílovém vrcholu na směr doleva. Protože však tato změna není pro důkazy složitosti tango stromu nutná, rozhodli jsme se jí nedělat (naopak změna preferovaného směru v cílovém vrcholu multisplay stromu nutná je, tu tedy provádíme).

Všechny ostatní implementační problémy řešíme přímočaře přesně podle popisu v první kapitole této práce a příslušných článků.

\section{Měření}

Měření počtu dotčených vrcholů budeme provádět tak, že každé operaci \ope{Find} přiřadíme její pořadové číslo. V každém vrcholu si zaznamenáme pořadové číslo operace, při které byl naposled navštíven. Poté vždy, když se dotkneme daného vrcholu, kde dotykem myslíme dereferencování na něj mířícího ukazatele, zkontrolujeme, zda jsme ho během této operace již viděli, a pokud ne, změníme číslo poslední operace, které máme v tomto vrcholu zaznamenané, a zvýšíme počítadlo dotčených vrcholů o jedna. 
