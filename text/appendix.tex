\section{Dokumentace}

Součástí elektronických příloh práce je zdrojový kód v~jazyce C programů použitých k~měření vlastností stromů. Zde poskytneme dokumentaci tohoto zdrojového kódu.

\catcode`_=\active
\def _{\textunderscore}

\catcode`!=\active
\def!#1!{{\tt #1}}



Kód je rozdělen do celkem 9 souborů. Kód v~souboru {\tt stack.h} obsahuje naši
implementaci zásobníku. Rozhodli jsme se totiž v implementaci operací \ope{Find} nepoužít žádnou knhovní funkci, abychom měli kontrolu nad zacházením s cachí. Zásobník předpokládá, že před jeho vložením do zdrojového kódu budou nadefinována následující objektová makra, jejichž expanzí budou validní identifikátory jazyka C:

\begin{itemize}
\item !STACKNAME!
\item !STACKTYPE!
\item !INIT!
\item !DESTROY!
\item !PUSH!
\item !POP!
\item (volitelně) !PEEK!
\end{itemize}

Tím budou nadefinovány následující typy a funkce (makrem v~tomto popisu máme na mysli jeho hodnotu po expanzi):
\begin{itemize}
\item datový typ !STACKNAME!, který odpovídá stejnojmenné struktuře,
\item funkce !void INIT(STACKNAME *stack, size_t initialsize)!, která inicializuje zásobník,
\item funkce !void DESTROY(STACKNAME *stack)!, která uvolní všechny zdroje zásobníku,
\item funkce !void PUSH(STACKNAME *stack, STACKTYPE value)!, která vloží nový prvek do zásobníku,
\item funkce !STACKTYPE POP(STACKNAME *stack)!, která odstraní poslední vloženou hodnotu ze zásobníku a vrátí ji,
\item funkce !STACKTYPE PEEK(STACKNAME *stack)!, která vrací poslední hodnotu vloženou do zásobníku, ale neodstraňuje ji,
\end{itemize}

Dále zásobník obsahuje prvky !stack.used! a !stack.size!, které obsahují
aktuální počet prvků v~zásobníku a aktuální kapacitu zásobníku. Vnitřně
zásobník funguje tak, že pokud by měl počet prvků v~zásobníku překročit
kapacitu, je kapacita zdvojnásobena. Kapacitu nikdy nezmenšujeme, protože to
pro naši aplikaci zásobníku není potřeba.

Kód v~souboru !random.c! obsahuje implementaci pseudonáhodného generátoru
!xoshiro256**!, který představili \citet{xoshiro}. Tento generátor potřebuje 256-bitový seed. V~našich aplikacích
používáme pevný 256-bitový seed, a v~případě, že potřebujeme více různých
náhodných posloupností, použijeme opakované volání funkce !jump!, kterou tento
pseudonáhodný generátor implementuje. Jedno zavolání funkce !jump! je
ekvivalentní $2^{64}$ volání funkce !next!, kterou jinak používáme na generování náhodných čísel. Volání funkce !jump! je pro tento
generátor doporučený způsob generování více nezávislých posloupností náhodných
čísel.

Soubor {\tt common.c} obsahuje kód společný více
stromům -- definici struktury vrcholu, počítadlo doteků, rotace a alokace vrcholů a kvůli
tomu, že tango strom interně využívá červenočerné stromy, také většinu kódu
specifického pro červenočerný strom (operace \ope{Join}, \ope{Split} a kód
starající se o~vyvažování po operaci \ope{Insert}). Naopak neobsahuje kód
potřebný pro vykonání operace \ope{Splay}, protože po této operaci chceme
v~splay stromu a v~multisplay stromu něco trochu jiného a došli jsme k~názoru, že
bude jednodušší nechat tyto implementace oddělené. Podobně jsme zůstali
u~separátních implementací funkcí vypisující tvar stromů na standardní výstup pro
účely ladění.

Soubory {\tt rb.c}, {\tt splay.c}, {\tt multisplay.c} a {\tt tango.c} obsahují
implementace jednotlivých stromů. Tyto soubory je možné samostatně přeložit --
tím dostaneme program, který neakceptuje žádný vstup a při spuštění krátce
demonstruje funkčnost daného stromu. Všechny čtyři programy také ukončí svůj
chod vypsáním počtu bytů, které zabírá v~paměti jeden vrchol tohoto stromu.

Tyto soubory lze také použít jako moduly do jiného programu -- pomocí inkludování příslušného souboru. Při takovém použití je nutné nadefinovat příslušné z~maker
!SPLAY!, !REDBLACK!, !MULTISPLAY! nebo !TANGO!, čímž docílíme, že preprocesor
odstraní funkci !main!. Dále je možné nadefinovat makro !VALUE!, jehož expanzí
bude datový typ, který chceme do stromu ukládat. Tento datový typ musí
podporovat porovnávání pomocí operátoru !>!. Výchozí hodnota tohoto makra je
!int!. Pokud makro nadefinujeme, je dále nutné nadefinovat makro
!VALUE_FORMAT!, jehož expanzí bude formátovací řetězec odpovídající typu
!VALUE!. Dále je možné nadefinovat makro !INVALID!, jehož expanzí bude hodnota
typu !VALUE!, která do stromu nesmí být nikdy vložena. Ta je pak interně
používána jako chybová hodnota a je také vrácena při hledání v~prázdném stromu.

Inkludování souboru nám nadefinuje datový typ !Node!, do nějž lze ukládat strom. Dále nám nadefinuje několik funkcí pro práci se stromem, konkrétně:

\begin{itemize}
\item !void print_tree(Node *root)!, která vytiskne textovou reprezentaci stromu na standardní výstup,
\item v~případě multisplay a tango stromu !Node *build(int hi)!, která vytvoří strom na vrcholech 0 až !hi!,
\item v~případě multisplay a tango stromu !VALUE *find(VALUE value, Node **root)!, která vyhledá ve stromu hodnotu !value! a vrátí ji, pokud ve stromu je, případně jejího předchůdce či následníka ve stromu, pokud !value! ve stromu není,
\item v~případě červenočerného stromu !VALUE *find(VALUE value, Node **root, bool insert)!, která vyhledá ve stromu hodnotu !value! a vrátí ji, pokud ve stromu je, jinak ji do stromu vloží a vrátí ji, pokud nastavíme příznak !insert!, případně jejího předchůdce či následníka ve stromu, pokud !value! ve stromu není a !insert! je !false!. Na adrese !*root! potom vrátí červenočerný strom po změně, kterou v něm mohl !find! s příznakem !insert! udělat.
\item v~případě splay stromu !VALUE *splay(VALUE value, Node **root, bool insert)!, která funguje stejně jako funkce !find! červenočerného stromu,
\item v~případě červenočerného stromu funkci !void split(Node **root, VALUE value)!, která vrátí na adrese !*root! strom, jehož kořen je vrchol obsahující hodnotu !value! a jeho levý a pravý podstrom jsou validní červenočerné stromy obsahující dohromady přesně vrcholy z~vstupního stromu krom vrcholu s~hodnotou !value!,
\item v~případě červenočerného stromu funkci !void join(Node **root)! inverzní k~funkci !split!.
\end{itemize}

Soubor {\tt run_tests.c} obsahuje kód nutný k~provedení měření. Kromě měření
samotného se jedná o~implementaci operace \ope{Build} pro červenočerný a splay
strom, které ve svých vlastních souborech implementují pouze operaci
\ope{Insert}. Tento soubor je také možné přeložit. Při překládání očekává, že
bude předem nadefinováno právě jedno z~maker !SPLAY!, !REDBLACK!, !MULTISPLAY!
nebo !TANGO!, podle čehož vybere, který strom bude měřit. Dále reaguje ještě na
dvě další makra. Prvním z~nich je makro !TOUCH!, které vede k~tomu, že bude
přeložený program měřit počet dotyků (jinak bude měřit čas procesu). Druhým je
potom makro !RANDOM_ALLOCATION!, které vede k~tomu, že jsou adresy nových
vrcholů přiřazovány náhodně (jinak jsou přiřazovány v~rostoucím pořadí).

Na tato dvě makra ve skutečnosti reagují i samy jednotlivé stromy. Při jejich
překladu se ale makro !TOUCH! projeví pouze větší velikostí vrcholů, protože
proměnná, která udává pořadové číslo operace, není nikdy změněna, a proměnná,
v~níž se ukládá celkový počet doteků, není nikdy přečtena. Makro
!RANDOM_ALLOCATION! se potom vůbec neprojeví v~chování přeloženého
červenočerného a splay stromu. Při jejich běhu se totiž nikdy nezavolá
inicializační funkce alokátoru paměti a nové vrcholy se alokují pomocí funkce
!malloc!.

Po přeložení zdrojového souboru !run_tests.c! dostaneme program, který po
spuštění očekává na prvním řádku svého standardního vstupu celé číslo
z~intervalu $[0, \text{!MAX_INT!}]$ odpovídající požadované velikosti testovaného stromu. Dále
očekává na dalších řádcích vstupu přístupovou posloupnost, a to tak, že každý
řádek standardního vstupu bude obsahovat právě jedno číslo. Na konci
standardního vstupu program vypíše buď počet dotyků vrcholu, nebo celkový
počet milisekund strávených operacemi \ope{Find}.

A~nakonec soubor !data.c! obsahuje generátor dat. Tento soubor po přiložení vydá program, který na příkazové řádce očekává 4 až 5 argumentů. Nejprve požadovanou velikost stromu $n$. Potom požadovanou délku přístupové posloupnosti $m$. Potom seed náhodného generátoru, u~nějž je nutné, aby se jednalo o~malé číslo, protože inicializace trvá lineárně dlouho vůči hodnotě seedu. Dále znak určující typ posloupnosti. Možné hodnoty jsou

\begin{itemize}
\item !s! pro sekvenční přístupovou posloupnost,
\item !r! pro náhodnou přístupovou posloupnost,
\item !i! pro bit reversal posloupnost levé páteře,
\item !u! pro podmnožinovou náhodnou posloupnost,
\item !b! pro podmnožinovou sekvenční posloupnost,
\item !e! pro proměnlivě podmnožinovou posloupnost.
\end{itemize}

Pro podmnožinové posloupnosti program očekává ještě pátý argument s~požadovanou velikostí podmnožiny. Všechny číselné argumenty kromě seedu a délky posloupnosti musí být kladná celá čísla, seed a délka posloupnosti mohou být i nula.

\catcode`_=8

\openright
\section{Proměnlivě podmnožinová posloupnost}\label{sec:app2}
{\def\dira#1{}
\doublegraphfigure{rb}e{Červenočerný strom -- proměnlivě podmnožinová posloupnost}
}
\doublegraphfigure{splay}e{Splay strom -- proměnlivě podmnožinová posloupnost}
\doublegraphfigure{multisplay}e{Multisplay strom -- proměnlivě podmnožinová posloupnost}
\doublegraphfigure{tango}e{Tango strom -- proměnlivě podmnožinová posloupnost}

