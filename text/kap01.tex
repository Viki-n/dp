\chapter{Teoretický úvod}

V této kapitole zavedeme formální definici binárního vyhledávacího stromu. Poté
se podíváme na třídu rankově vyvážených stromů a jejich speciální případ Weak
AVL strom. Potom představíme několik mezí optimality, vlastností, které musí
mít optimální binární vyhledávací strom. Nakonec se podíváme na Tango stromy a
Multisplay stromy a dokážeme, že tyto meze splňují (nebo, v případě tango
stromu, že je téměř splňují).

\section{Výpočetní model}
Chceme-li mluvit o optimálním stromu, musíme nejprve specifikovat, co přesně
budeme za binární vyhledávací strom považovat. Definici, kterou zde
představíme, používali implicitně už \citet{splay}, formalizoval ji však až
\citet{tango}.

\begin{definice}
Mějme statické \emph{univerzum} klíčů $\mathcal U = \{1,2,\dots,n\}$. Dále
mějme binární strom nad těmito klíči takový, že pro každé dva vrcholy $v_1$,
$v_2$ platí, že je-li vrchol $v_1$ v podstromu vrcholu $v_2$, pak je v jeho
levém podstromu, právě když je jeho klíč menší než líč vrcholu $v_2$. Nakonec
mějme \emph{přístupovou posloupnost} $x_1,x_2,\dots,x_m$, kde $\forall i \in
\mathbb N, i\leq m$ platí $x_i\in \mathcal U$. Pak \emph{přístupový algoritmus
binárního vyhledávacího stromu} je algoritmus, který postupně provede přístupy
ke vrcholům s klíči $x_1, x_2,\dots,x_m$.

Přístup probíhá tak, že algoritmus smí mít vždy právě jeden ukazatel na vrchol
stromu, který na počátku každého přístupu ukazuje na kořen stromu. Dále v
každém kroku smí provést právě jednu z následujících operací: 
\begin{itemize}
\item Přesunout ukazatel na levého syna aktuálního vrcholu,
\item přesunout ukazatel na pravého syna aktuálního vrcholu,
\item přesunout ukazatel na rodiče aktuálního vrcholu,
\item provést rotaci hrany mezi aktuálním vrcholem a jeho rodičem.
\end{itemize}

Řekneme, že čas běhu algoritmu je počet těchto operací, které za sekvenci
přístupů provede, plus jedna. O vrcholu stromu řekneme, že jsme se ho při daném
přístupu \emph{dotkli}, pokud na něj někdy během tohoto přístupu ukazoval
ukazatel algoritmu.  \end{definice}

Takovému přístupovému algoritmu se někdy také říká \emph{offline přístupový
algoritmus}. V praxi ale potřebujeme přístupy provádět online.

\begin{definice}
\emph{Online přístupový algoritmus} je tekový přístupový algoritmus, jehož
rozhodnutí během $i$-tého přístupu nijak neovlivňují hodnoty $x_j$ z přístupové
posloupnosti pro $j>i$. Na druhou stranu si tento algoritmus smí v každém
vrcholu uložit až $\mathcal O(1)$ slov paměti informací (nikoli však ukazatele
na vrcholy).  
\end{definice}

Všimneme si, že běžné algoritmy binárních vyhledávacích stromů tuto definici
splňují -- Například červenočerné a AVL stromy potřebují v každém vrcholu
jediný bit informace, splay strom se obejde zcela bez dalších informací.

Pro danou přístupovou sekvenci $X$ existuje přístupový algoritmus, který ji
vykoná optimálně, tedy v nejkratším čase ze všech možných algoritmů. Tento
počet operací označíme $\opt(X)$. Zde předpokládáme, že je strom na začátku v
nejlepší možné konfiguraci. Tím však nesnížíme potřebný čas na přístupy o více
než aditivní $\mathcal O(n)$, protože z libovolného BVS je možné pomocí
$\mathcal O(n)$ rotací vytvořit libovolný jiný (nad tou samou množinou klíčů).
Proto budeme dále zkoumat pouze přístupové posloupnosti $X$ takové, že $|X| \in
\Omega(n)$. Vzhledem k tomu, že nahlédneme, že $\opt(x)\geq |X|$, je tento
faktor asymptoticky zandebatelný. 


\begin{definice}
O přístupovém algoritmu řekneme, že je \emph{$f(n)$-kompetitivní}, pokud každou
posloupnost přístupů $X$ nad univerzem velikosti $n$ vykoná v čase $\mathcal
f(n)\cdot\opt(X)$. O online přístupovém algoritmu řekneme, že je
\emph{dynamicky optimální}, pokud je $\mathcal O(1)$-kompetitivní.
\end{definice}

Předtím, než si popíšeme některé konkrétní stromy, už pouze podotkneme, že
existence dynamicky optimálního algoritmu je otevřeným problémem. Na druhou
stranu například o splay stromech vyslovili \citet{splay} hypotézu, že jsou
dynamicky optimální.

\section{Rankově vyvážené stromy}

\section{Meze optimality}

\section{Tango stromy}

\section{Multisplay stromy}
