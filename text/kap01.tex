\chapter{Teoretický úvod}

V této kapitole zavedeme formální definici binárního vyhledávacího stromu a
pojednáme o jeho výhodách oproti jiným způsobům uložení dat. Dále přiblížíme
(staticky) optimální strom. Pro dynamické stromy připomeneme některé běžné
vyvažovací strategie a také představíme jiné, novější přístupy, například třídu
rankově vyvážených stromů a jejich speciální případ Weak AVL strom. Potom
představíme několik mezí optimality, vlastností, které musí mít optimální
binární vyhledávací strom. Nakonec se podíváme na Tango stromy a Multisplay
stromy a dokážeme, že tyto meze splňují (nebo, v případě tango stromu, že je
téměř splňují).

\section{Binární vyhledávací strom}
\def\U{\mathcal U}
\def\o{\mathcal O}
\let\op\operatorname

Častým úkolem datové struktury je udržovat informaci o množině prvků. Formálně
mějme univerzum $\mathcal U$, lineární uspořádání na tomto univerzu $\leq$, a
konečnou podmnožinu tohoto univerza $M$. Budeme chtít vybudovat datovou strukturu, která bude umět o
každém prvku $\mathcal U$ říct, jestli náleží množině $M$. Po takových
strukturách často chceme nějaké z následujících operací:

\begin{itemize}
\item $\op{Find}(x)$: pro prvek $x \in \U$ rozhodnout, zda $x\in M$,
\item $\op{Insert}(x)$: přidat prvek $x \in \U$ do množiny $M$,
\item $\op{Delete}(x)$: odstranit prvek $x\in M$ z $M$,
\item $\op{Enumerate}()$: vyjmenovat všechny prvky $M$,
\item $\op{Succ}(x)$, resp $\op{Pred}(x)$: pro daný prvek $x\in \U $ najít nejbližší (ostře) větší, resp. menší prvek $M$,
\item $\op{RangeQuery}(x,y)$: pro dané dva prvky $x,y\in\U$ takové, že $x<y$, spočítat nebo vyjmenovat všechny prvky $z\in M$ takové, že $x \leq z \leq y$.
\end{itemize}

Volba struktury v každém konkrétním případě závisí na tom, které z těchto
operací potřebujeme v naší konkrétní aplikaci podporovat. Pokud nám například
stačí umět prvky vyjmenovat, postačí nám obyčejné pole. Pokud chceme umět
vyhledávat, přidávat i mazat v průměru v konstantním čase, je nejlepší volbou nějaká
varianta hashovací tabulky. Zejména poslední tři zmíněné operace ale hashovací
tabulka umí pouze v čase $\Theta(|M|)$. Pokud tedy tento typ dotazů potřebujeme
struktuře pokládat, můžeme využít nějakou ze stromových datových struktur,
které v této práci popíšeme.

\begin{definice}
Mějme datovou strukturu, která se skládá z vrcholů. Každý vrchol může mít až
dva následníky, kterým budeme říkat \emph{levý} a \emph{pravý syn}. Jeden
vrchol nazveme \emph{kořen}. Všechny ostatní vrcholy pak budou mít právě
jednoho \emph{rodiče} (tak, že pokud vrchol $u$ je rodičem vrcholu $v$, pak
vrchol $v$ je pravým nebo levý synem vrcholu $u$)\footnote{V některých
implementacích si vrcholy nemusí pamatovat ukazatele na své rodiče, my budeme
však prozatím předpokládat, že si je pamatují.}. Pro vrchol $v$ nazveme
\emph{podstrom vrcholu $v$} množinu vrcholů, do nichž se dá dostat z z vrcholu
$v$ libovolnou\footnote{Posloupnost může být i prázdná.} posloupností kroků 
\uv{přejdi do pravého syna} a \uv{přejdi do levého syna}.

V každém vrcholu bude pak právě jedna hodnota, a navíc bude pro každý vrchol $v$ platit, že hodnoty ve všech vrcholech v podstromu levého syna $v$ jsou ostře menší než hodnota ve $v$, a hodnoty v podstromu pravého syna naopak ostře větší.

Nakonec si zapamatujeme ukazatel na kořen tak, abychom ho kdykoli v konstantním čase nalezli.

Této datové struktuře budeme říkat \emph{binární vyhledávací strom}
\end{definice}

Při práci se stromem budeme předpokládat, že následující kroky umíme v konstantním čase:
\begin{itemize}
\item Nalezni kořen,
\item přejdi do pravého, případně levého syna aktuálního vrcholu,
\item přejdi do rodiče aktuálního vrcholu,
\item přečti či zapiš hodnotu v aktuálním vrcholu,
\item pokud aktuální vrchol nemá pravého či levého syna, vytvoř na jeho místě nový vrchol s danou hodnotou.
\end{itemize}
Vyhledávání hodnoty $x$ ve stromu pak můžeme provést podle následujícího algoritmu:
\begin{enumerate}
\item Nalezni kořen.
\item Podívej se, zda je hodnota $x$ rovna hodnotě v aktuálním vrcholu, a pokud ano, nahlaš nález.
\item Je-li hodnota $x$ větší než hodnota v aktuálním vrcholu, podívej se, zda existuje pravý syn aktuálního vrcholu. Pokud ano, přejdi do něj a vrať se ke 2. kroku algoritmu, jinak ohlaš, že  $x$ ve stromu není.
\item Je-li hodnota $x$ menší než hodnota v aktuálním vrcholu, podívej se, zda existuje levý syn aktuálního vrcholu. Pokud ano, přejdi do něj a vrať se ke 2. kroku algoritmu, jinak ohlaš, že  $x$ ve stromu není.
\end{enumerate}

Podobným algoritmem budeme prvky přidávat, pouze místo nahlášení neexistence na místě chybějícího syna založíme nový vrchol. Mazání prvku provedeme následovně:

\begin{itemize}
\item Je-li vrchol obsahující mazaný prvek list, prostě ho smažeme.
\item Má-li vrchol obsahující mazaný prvek právě jednoho syna, prvek smažeme a jeho syna připojíme pod rodiče mazaného prvku na místo mazaného prvku.
\item Má-li vrchol obsahující mazaný prvek dva syny, najdeme ve stromě prvek nejblíže větší, jeho
hodnotu zapíšeme na místo odstraňovaného prvku a odstraníme vrchol, v němž byl tento nejblíže větší
prvek, kterému nutně chybí přinejmenším levý syn (kdyby existoval, hodnota v
něm by byla menší než hodnota jeho rodiče, ale větší než hodnota odstraňovaného
vrcholu -- tedy jeho rodič by nemohl být nejbližší větší).  
\end{itemize} 

\section{Statická optimalita}

Podívejme se na životní cyklus binárního vyhledávacího stromu. Takový strom
nejprve vybudujeme s (ne nutně neprázdnou) počáteční množinou klíčů
a potom na na něm vykonáme nějakou posloupnost operací. My se prozatím omezíme
pouze na případ, kdy strom vybudujeme nad neprázdnou množinou klíčů, a jediná
operace, kterou budeme vykonávat, bude operace vyhledání. Navíc budeme
požadovat, aby všechny klíče, které ve stromě hledáme, skutečně ve stromě byly.
Potom je posloupnost operací jednoznačně dána pouze hodnotami, na které se
dotazujeme (a jejich pořadím; hodnoty nemusí být různé). Posloupnost těchto
hodnot budeme nazývat \emph{přístupová posloupnost}, budeme ji značit $S$ a
její délku budeme značit $m$.

Přirozená otázka zní, jak vypadá binární vyhledávací strom, který pro danou
posloupnost přístupů vykoná nejméně operací. V plné obecnosti budeme tuto otázku řešit v pozdějších částech této kapitoly. Pro tuto sekci se omezíme na statické
stromy. To znamená stromy, které vybudujeme nad předem danou množinou klíčů, a
za běhu už nebudeme podporovat vkládání a mazání, ani neumožníme měnit
strukturu stromu. Na druhou stranu budeme předpokládat, že přístupovou posloupnost, nebo alespoň četnost jednotlivých prvků v ní, předem známe. V tomto modelu se počet operací přesně rovná
počtu navštívených vrcholů (včetně násobnosti).

Problém konstrukce optimálního stromu v tomto modelu vyřešil dynamickým
programováním \citet{staticoptimality}. Myšlenku jeho algoritmu zde představíme.

Mějme množinu klíčů, pro jednoduchost {\tt [n]}\footnote{Tímto značení myslíme
množinu přirozených čísel od 1 do $n$ včetně.} a pole vah {\tt w}. Váha vrcholu
{\tt i} {\tt w[i]} může být buď pravděpodobnost, že daný přístup bude mít za
cíl naleznout vrchol {\tt i}, nebo počet výskytů {\tt i} v přístupové
posloupnosti. V prvním případě dostaneme jako váhu výsledného stromu střední
hodnotu počtu navštívených vrcholů při přístupu, v druhém případě součet počtu operací přes
celou přístupovou posloupnost.

Algoritmus je založený na myšlence, že podstrom libovolného vrcholu v
optimálním stromu je sám o sobě optimální strom nad příslušnými prvky. Proto
stačí pro všechny možné volby kořene rekurzivně spočítat optimální strom pro
prvky menší než kořen a větší než kořen, a z takto spočítaných stromů vybrat
ten s nejmenší váhou. Protože přímočará implementace tohoto postupu by běžela v
exponenciálním čase, budeme si váhy (a případně kořeny) dílčích optimálních
stromů ukládat. Naivní (a suboptimální) implementace tohoto postupu vypadá
takto:

\code{
\l GetStaticallyOptimalTreeWeight(1, n, w):
\block
\l TreeWeights = Array[n + 1, n + 1]
\l \# V buňce i, j bude váha optimálního stromu 
\l \# na vrcholech i až i+j-1
\l WeightSums = Array[n + 1, n + 1]
\l \# V buňce i, j bude součet vah i-tého až j-1. vrcholu
\l for i in range(1, n + 2)
\block
\l for j in range(1, n + 2)
\block
\l WeightSums[i, j] = sum(w[i:j])
\endblock
\endblock
\l for Size in range(1, n + 1)
\block
\l for Start in range(1, n + 2 - Size)
\l End = Start + Size - 1
\block
\l Weight = $\infty$
\l for Root in range(start, start + size)
\block
\l LeftSubtreeWeight = \b TreeWeights[Start, Root - 1 - Start] \b + WeightSums[Start, Root]
\l RightSubtreeWeight = \b TreeWeights[Root + 1, End - Root] \b + WeightSums[Root + 1, End + 1]
\l Weight = min(Weight, w[Root] \b + LeftSubtreeWeight \b + RightSubtreeWeight) \refline{prirazovanivah}
\endblock
\l TreeWeights[Start, Size] = Weight
\endblock
\endblock
\l return TreeWeights[1, n]
\endblock
}

Kdyby nás zajímala nejen váha optimálního stromu, ale chtěli bychom ho skutečně
zkonstruovat, bylo by nutné na řádku \codelineref{prirazovanivah} a
následujícím ukládat nejen minimum, ale i tvar stromu na vrcholech {\tt Start}
až {\tt End}jako tvar optimálního levého a pravého podstromu s vrcholem {\tt
Root} jako kořenem. Výpočet obsahu pole {\tt WeightSums} by samozřejmě šel
udělat i kvadraticky s využitím prefixových součtů, nebo bychom ho vůbec
nemuseli počítat a mohli bychom si nechat pouze pole prefixových součtů.
Protože ale algoritmus tak, jak jsme ho představili, má tak jako tak kvadratickou prostorovou a kubickou
časovou složitost, není z asymptotického pohledu nutné tuto část algoritmu
optimalizovat. Algoritmus však lze upravit tak, aby jeho časová složitost byla pouze kvadratická.


\section{Vyvažované stromy}

Co když však potřebujeme umět do struktury i vkládat? Můžeme pokračovat naivně
podle algoritmu uvedeného výše v této kapitole. \citet{sortingsearching}
ukázal, že pokud budeme prvky do stromu vkládat v náhodném pořadí\footnote{Bez
újmy na obecnosti předpokládáme, že do struktury postupně vložíme všechny prvky
z množiny $[n]$ pro vhodně zvolené $n$. Potom náhodné pořadí znamená pořadí
určené rovnoměrně náhodně zvolenou permutací z množiny $S_n$}, s vysokou
pravděpodobností dostaneme strom, v němž bude průměrná hloubka vrcholu $2 \ln n$.
\citet{Robson} dále ukázal, že pokud $H(n)$ střední hodnota hloubky nejhlubšího 
vrcholu ve stromě o $n$ vrcholech, pak $$\lim_{m\rightarrow
\infty}\frac{H(n)}{\ln(n)}\leq\alpha,$$ kde $\alpha$ je přibližně
$4.311\dots$, přesně se jedná o největší kořen rovnice $$\alpha\cdot \ln
\frac{2e}{\alpha} = 1.$$ \citet{devroye} později dokázal, že výše uvedená
nerovnost je ve skutečnosti rovnost.

V praxi však často klíče potřebujeme vkládat i v pořadí, které není náhodné,
ale obsahuje nějaký vzorec. Například pokud bychom do (na počátku prázdného)
nevyvažovaného binárního stromu vložili čísla z množiny $[n]$ vzestupně, strom
zdegeneruje v jednu jedinou cestu, na níž bude nejhlubší vrchol v hloubce $n$ a
průměrná hloubka vrcholu bude $(n+1)/2$. Takovýmto případům bychom se rádi
vyhnuli, proto prozkoumáme vyvažované stromy. Protože však všechny stromy,
které budeme zkoumat, budou vyvažované pomocí rotací hran, nejprve představíme,
co taková rotace hrany je.

\begin{definice}
Mějme vrchol binárního vyhledávacího stromu $u$ a jeho (bez újmy na obecnosti levého) syna $v$. Dále si
označíme $p$ rodiče vrcholu $u$, $A$ a $B$ popořadě levý a pravý podstrom $v$ a
$C$ pravý podstrom $u$. Pak \emph{(jednoduchá) rotace hrany $uv$} je krok,
při níž nastavíme vrchol $v$ jako syna $p$ (na stejné straně, kde byl původně
vrchol $u$), vrchol $u$ jako pravého syna $v$ a podstrom $B$ jako levý podstrom
$u$. Podstromy $A$ a $C$ zůstanou připojené k vrcholům $v$ a $u$ tak, jak byly.

Dále mějme vrchol $u$, jeho (bez újmy na obecnosti levého) syna $v$ a syna
vrcholu $v$, kterého si označíme $w$. Potom pokud $w$ je pravý syn, můžeme
provést \emph{dvojitou rotaci hran $wv$ a $vu$ typu zig-zag}, která probíhá tak,
že nejprve zrotujeme hranu $wv$, a potom hranu $vu$. Pokud je $w$ levý syn,
můžeme provést \emph{dvojitou rotaci hran $wv$ a $vu$ typu zig-zig}, která
probíhá tak, že nejprve zrotujeme hranu $vu$ a potom hranu $wv$.
\end{definice}

Rotace hrany se může zdát jako zvláštní operace. Téměř všechny vyvažovací algoritmy (a úplně všechny, o kterých zde budeme mluvit) ale využívají k vyvažování právě rotace, protože se jedná o poměrně jednoduchou a překvapivě silnou operaci, která zachovává uspořádání.

Dvojité rotace typu zigzig budeme potřebovat později v této kapitole. V této
sekci si vystačíme s jednoduchými rotacemi a dvojitými rotacemi typu zig-zag,
proto budeme-li mluvit o dvojitých rotacích, budeme tím mít na mysli právě typ
zigzag.

\subsection{AVL stromy}

AVL strom poprvé představili \citet{AVL}. Jedná se o typ vyvažovaného stromu. V
AVL stromu platí, že hloubka pravého a levého podstromu\footnote{Pravým, resp. levým podstromem vrcholu myslíme podstrom pravého, resp. levého syna tohoto vrcholu} každého vrcholu se liší
nejvýše o jedna. To znamená, že pokud $D(n)$ označíme minimální počet vrcholů
ve stromu hloubky $n$, dostáváme, že $D(0)=0$, $D(1)=1$ a
$D(i)=1+D(i-1)+D(i-2)$ pro každé $i$ větší než jedna. Pokud vyřešíme tuto
rekurenci, zjistíme, že $D(i)=\log_\varphi(i) + \o(1)$, kde $\varphi = 1.618\dots$ je zlatý
řez.

V každém vrcholu si budeme pamatovat navíc informaci o vyvážení tohoto vrcholu.
Tato informace může nabývat tří možných hodnot -- daný vrchol může být buď
\emph{vyvážený} (oba jeho podstromy jsou stejně hluboké), nebo \emph{nakloněný
doprava}, (jeho pravý podstrom je o jedna hlubší než levý), nebo
\emph{nakloněný doleva}\footnote{Vystačili bychom si i s jediným bitem, který by udával, zda je podstrom daného vrcholu hlubší než podstrom jeho souseda.}.

Vyhledání prvku probíhá stejně, jako v nevyvažovaném binárním vyhledávacím stromě. 
Vkládání je ale trochu komplikovanější. Při připojení nového listu je potřeba
jeho rodiči změnit vyvážení. Byl-li vyvážený (tj. byl to list), nově bude
nakloněný za novým vrcholem. Byl-li nakloněný od nového vrcholu, bude nově
vyvážený. V prvním případě navíc vzrostla celková hloubka podstromu tohoto
vrcholu, je tedy potřeba změnu vyvážení propagovat výše do stromu. Kromě
předchozích dvou případů  může při další propagaci nastat ještě jeden jiný,
totiž že byl vrchol už původně nakloněný směrem ze kterého propagujeme. To
znamená, že po přidání nového vrcholu již neplatí invariant AVL stromu a musíme
provést (v závislosti na konkrétní situaci buď jednoduchou nebo dvojitou, viz
obrázek TADY BUDE REFERENCE) rotaci hrany, abychom situaci napravili.
Provedením rotace však zařídíme, že se celý podstrom rotovaného vrcholu o jedna
sníží. To znamená, že bude mít stejnou výšku jako před insertem a změnu tedy
není potřeba dále propagovat.   

Následující pseudokód předpokládá neprázdný
AVL strom -- pro prázdný strom stačí vytvořit nový vrchol, nastavit ho jako vyvážený a namířit na něj ukazatel kořene.


\code{
\l AVLinsert(Root, Element)
\block
\l Current = Root
\l Next = Current.value > Element ? Current.leftSon : Current.rightSon
\l while Next != NULL
\block
\l Current = Next
\l Next = Current.value > Element ? Current.leftSon : Current.rightSon
\endblock
\l New = Node(Element)
\l New.balance = VYVÁŽENÝ
\l New.parent = Current
\l if Current.value > Element
\block
\l Current.leftSon = New
\endblock
\l else
\block
\l Current.rightSon = New
\endblock
\l
\l Current = New
\l Next = Current.parent
\l while Next != NULL
\block
\l ComingFromRight = Current == Next.rightSon
\l if Next.balance == VYVÁŽENÝ
\block
\l Next.balance = ComingFromRight ? NAKLONĚNÝ DOPRAVA : NAKLONĚNÝ DOLEVA
\endblock
\l else if (Next.balance == NAKLONĚNÝ DOLEVA) == ComingFromRight
\block
\l Next.balance = VYVÁŽENÝ
\l return
\endblock
\l else
\block
\l
\l TADY JEŠTĚ KUS CHYBÍ
\l
\endblock
\l Current = Next
\l Next = Current.parent
}

Operaci delete zde podobně detailně probírat nebudeme, podotkneme ale, že na rozdíl od operace insert může být nutné až řádově logaritmicky mnoho rotací vůči vůči počtu vrcholů ve stromě (tj. lineárně mnoho vůči hloubce stromu).



\subsection{Červenočerný strom}

Červenočerný strom odvodili z 2-4 stromů \citet{redblack}. Pro popis
červenočerného stromu se nám bude hodit alternativní pohled na binární stromy.
Představíme si, že v binárním stromu bude každý vrchol mít buď právě dva, nebo
žádného potomka. Vrcholy, které nemají žádného potomka, neobsahují žádný klíč
ani jiná data. V programu tedy mohou být reprezentovány například nulovými
ukazateli. Těmto vrcholům budeme říkat \emph{vnější} nebo také \emph{externí}
vrcholy.

Červenočerný strom je vyvažovaný binární strom, který dále splňuje následující invarianty:

\begin{enumerate}
\item Každý vrchol má buď červenou, nebo černou barvu.
\item Všechny externí vrcholy považujeme za černé.
\item Na každé cestě z kořene do externího listu musí být stejný počet černých vrcholů.
\item Každý červený vrchol má oba potomky černé.
\end{enumerate}

Některé zdroje dále uvádí, že kořen musí být černý, to však není problém kdykoli zařídit -- máme-li korektní červenočerný strom s červeným kořenem, můžeme kořen přebarvit na černo a všechny invarianty zůstanou splněny. Pokud však budeme černou barvu kořene požadovat, bude ve stromě lépe vidět souvislost s původními 2-4 stromy. V takovém případě si můžeme všimnout, že pokud zkontrahujeme všechny hrany mezi červeným synem černého otce, dostaneme strom, ve kterém má každý vrchol kromě listů mezi dvěma a čtyřmi potomky, a ve kterém jsou všechny cesty z kořene do listu stejně dlouhé.

Nyní se pokusíme odhadnout minimální počet vrcholů $D(n)$ ve stromu hloubky $n$. Strom s minimálním počtem vrcholů při dané hloubce vypadá tak, že obsahuje jednu cestu, na níž se střídají červené a černé vrcholy, a na každý její vrchol dále pověsíme dokonale vyvážený strom tvořený pouze černými vrcholy o hloubce určené tak, aby byly splněny invarianty. To znamená, že jeden z podstromů kořene je strom s minimálním počtem vrcholů při hloubce o jedna menší, a druhý je dokonale vyvážený strom. Kdybychom se pokusili $D(n)$ vyjádřit pouze pomocí $D(n-1)$, muselo by se nám ve vzorci objevit zaokrouhlování ve formě dolní celé části. Abychom se zaokrouhlování vyhnuli, vyjádříme $D(n)$ pomocí $D(n-2)$.

$D(n)$ můžeme vyjádřit\footnote{V následujícím výpočtu nebereme v úvahu externí vrcholy. Kdybychom je v úvahu brali, přesné vzorce by vypadaly trochu jinak, ale v závěru by se změna schovala v $\o$-notaci.} jako $D(1)=1$, $D(2) = 2$, $D(2i) = D(2i-2) + 2 \cdot
(2^{i - 1} - 1) + 2$, $D(2i + 1) = D(2i - 1) + 2^{i-1}-1 + 2^i-1 + 2$. Z toho
po zjednodušení a vyřešení dostáváme, že $D(i)=\Theta(2^{i/2})$, tedy nejvyšší
možná hloubka s daným počtem vrcholů je $H(n) = \log_{\sqrt 2} n + \o(1)$.
Protože  $\sqrt 2 < \varphi$, je hloubka červenočerného stromu v nejhorším
případě větší než hloubka AVL stromu v nejhorším případě při stejném počtu
vrcholů.

Průběh operací insert a delete zde nebudeme probírat, protože obsahují velké
množství speciálních případů, které by bylo potřeba popsat. Připojíme ale
dvě poznámky. První z nich je, že výhodou červenočerných stromů oproti AVL
stromům je vždy pouze konstantně mnoho změn struktury stromu při operaci. Druhá
věc, kterou je zde vhodné zmínit, je velice příbuzná struktura, tzv.
\emph{right-leaning červenočerný strom}, který představil \citet{rightleaning}.
V tomto stromu navíc platí, že pokud má vrchol právě jednoho červeného syna,
musí to být pravý syn. Tím sice může mírně vzrůst celkový počet rotací k
vyvážení struktury, ale výrazně se sníží počet případů, které je třeba při
vyvažování řešit a strom se tak stane implementačně výrazně jednodušší. V této
práci dále pokračoval \citet{leftleaning}, který dále zakázal vrcholy, jejichž
oba synové by byli červení. V této úpravě pak lze naimplementovat operaci
insert na pouhých 33 řádcích kódu v jazyce Java.    

Červenočerné stromy ale dále umí dvě operace, které zde představíme, protože
jich využívají datové struktury, které představíme později v této práci. Bude se jednat o operace \emph{join} a \emph{split}.
Před jejich popisem budeme ale potřebovat zavést ještě dva nové pojmy.

\begin{definice}
Mějme červenočerný strom $S$. Pak počet černých vrcholů na cestě mezi kořenem a (libovolným) externím vrcholem $S$ nazveme \emph{černou výškou} stromu $S$. Značit ji budeme $H'(S)$.
\end{definice}


\begin{definice}
Mějme libovolný neprázdný binární vyhledávací strom $S$ a jeho vrchol $u$, resp. $v$, odpovídající minimálnímu, resp. maximálnímu prvku v něm. Pak cestu z kořene do $u$, resp. $v$ nazveme \emph{levou}, resp. \emph{pravou páteří} stromu $S$.
\end{definice}

Operace join má na vstupu dva červenočerné
stromy $A$, $B$ a jeden další prvek univerza $x$. Pro tyto stromy musí platit,
že hodnoty všech prvků ve stromu $A$ jsou menší než $x$ a hodnoty všech prvků
ve stromu $B$ jsou větší než $x$. Na výstupu operace join máme jeden strom
obsahující všechny prvky, které obsahoval strom $A$, strom $B$, a prvek $x$. Za
předpokladu, že předem známe černou výšku stromů $A$
a $B$ $H'(A)$ a $H'(B)$, operaci join zvládneme v čase $\Theta(1+|H'(A) -
H'(B)|)$.

Při provádění operace join budeme postupovat následovně:

\begin{enumerate}
\item Pokud $H'(A) = H'(B)$, pak můžeme vytvořit nový černý vrchol s hodnotou $x$, jako jeho levého a pravého syna nastavit kořeny stromů $A$ a $B$ a vrátit ho jako kořen nového stromu.
\item Jinak bez újmy na obecnosti předpokládejme, že $H'(A)<H'(B)$. Potom můžeme jít po pravé páteři stromu $A$, dokud nenalezneme vrchol $v$ takový, že černá výška jeho podstromu je $H'(B)$. Vytvoříme nový červený vrchol $u$ s hodnotou $x$ a vložíme ho do stromu na místo vrcholu $v$. Vrchol $v$ připojíme pod $u$ jako levého syna a kořen stromu $B$ jako pravého syna. Mohlo se stát, že jsme právě rozbili invariant červenočerného stromu. Jeho platnost tedy obnovíme podobným postupem jako při normálním insertu.
\end{enumerate}

Druhá operace je operace split. Operace split dostane na vstupu strom\footnote{Z popisu operace split vyplyne, že není specifická pro červenočerné stromy. Tuto operaci můžeme provést pro libovolný typ binárního vyhledávacího stromu, který podporuje operace find a join. Pouze analýza časové složitosti bude specifická pro červenočerné stromy.} $S$ a $x\in \mathcal U$. Na výstupu má dva stromy $A$, $B$ takové, že všechny prvky ve stromu $A$ budou menší než $x$, a prvky ve stromu $B$ jsou větší než $x$. Stromy $A$ a $B$ navíc obsahují dohromady ty samé prvky, jako strom $S$ (kromě prvku $x$, pokud tento prvek byl ve stromě $S$). Operaci split zvládneme na červenočerném stromu v čase $\o(\log n)$. Postup bude následující:

\begin{enumerate}

\item Vyhledáme ve stromě $S$ prvek $x$. Pokud si v naší implementaci
červenočerného stromu nepamatujeme u každého vrcholu černou výšku jeho
podstromu, při cestě dolů budeme tyto černé výšky počítat. To můžeme udělat
například tak, že si ke každému vrcholu poznamenáme, kolik je černých vrcholů
na cestě mezi ním a kořenem (kořen včetně, vrchol sám nevčetně). Poté spočítáme
černou výšku celého stromu (pokud $x$ ve stromě není, stačí spočítat počet
černých vrcholů na cestě mezi kořenem a externím vrcholem, na jehož místo
bychom $x$ vkládali. Jinak pokračujeme stromem dolů až k libovolnému externímu
listu). Černá výška podstromů jednotlivých vrcholů je pak rozdíl mezi číslem,
které jsme si k nim poznamenali, a černou výškou původního
stromu\footnote{Černou výšku celého stromu ve skutečnosti ani počítat nemusíme.
Pro join ve skutečnosti nepotřebujeme znát černé výšky, stačí nám pouze umět v
konstantním čase určit rozdíl černých výšek dvou stromů, což lze již ze
zapamatovaných čísel.}.

\item Na cestě, kterou jsme při vyhledání prošli, odstraníme všechny hrany.
Vrcholy si roztřídíme do dvou seznamů podle toho, zda jsou větší nebo menší než
$x$. V seznamech udržujeme pořadí, v jakém jsme vrcholy na cestě našli --
seznamy jsou setříděné podle černých výšek podstromů, které zůstaly viset na
vrcholech. Pokud $x$ byl ve stromu, odstraníme ho a na konec našich seznamů
připojíme vždy příslušného syna $x$.

\item Nyní tedy máme dva seznamy vrcholů, kde na každém z těchto vrcholů visí
právě jeden syn. Podstrom tohoto syna je vždy validní červenočerný strom, a
navíc známe jeho černou výšku. Výjimkou může být vždy poslední vrchol v seznamu
-- ten může mít oba syny, ve kterémžto případě již on sám je kořen korektního
červenočerného stromu. Navíc platí, že oba seznamy jsou setříděny podle černých výšek stromů (tyto černé výšky nemusí být po dvou různé). Dále platí, že v seznamu prvků větších než $x$ je vždy hodnota vrcholu menší, než hodnota všech prvků v jeho zbývajícím (pravém) podstromu, ale větší, než hodnota všech prvků (a jejich zbývajících podstromů), které jsou v seznamu dále než oni (protože se původně jednalo o části jejich levého podstromu). Nyní tedy pro každý seznam zvlášť:

\item Pokud poslední prvek seznamu není kořen korektního červenočerného stromu, udělej z něj korektní červenočerný strom. To znamená, že vezmeme červenočerný strom, který visí na prvku seznamu, a prvek, na kterém původně visel, do něj vložíme standardní operací insert.

\item Všechny prvky seznamu spojíme operací join. Budeme postupovat od konce, a to vždy tak, že vezmeme poslední prvek seznamu, což je validní červenočerný strom, syna předposledního prvku, což je také červenočerný strom, a předposlední prvek seznamu sám (či přesněji jeho hodnotu), což je prvek univerza hodnotou mezi dvěma zmíněnými stromy, a aplikujeme na tyto dva stromy a tento prvek operaci join. Výsledný strom opět připojíme na konec seznamu. 

\end{enumerate}

Úvodní nalezení $x$ zvládneme v čase $\o(\log n)$. Při slučování využijeme toho, že stromy byly původně seřazeny podle černé výšky a spojování tuto výšku změnilo nejvýše o konstantu (uvědomíme si, že vždy nejvýše dva po sobě jdoucí stromy mohly mít původně stejnou černou výšku, černé výšky tedy nemohou růst příliš), tedy součet všech rozdílů černých výšek spojovaných stromů je nejvýše výška původního stromu plus jejich počet krát konstanta. Oba sčítance jsou nejvýše $\o(\log n)$, celá operace split tedy proběhne v čase $\o(\log n)$.

\subsection{Rank-balanced trees}


\section{Výpočetní model}
Chceme-li mluvit o optimálním stromu, musíme nejprve specifikovat, co přesně
budeme za binární vyhledávací strom považovat. Definici, kterou zde
představíme, používali implicitně už \citet{splay}, formalizoval ji však až
\citet{tango}.


\begin{definice}
Mějme \emph{přístupovou posloupnost} $S$ $x_1$,$x_2$,\dots,$x_m$, kde $\forall i \in
\mathbb N, i\leq m$ platí $x_i\in M$. Pak \emph{přístupový algoritmus
binárního vyhledávacího stromu} je algoritmus, který postupně provede přístupy
ke vrcholům s klíči $x_1, x_2,\dots,x_m$.

Přístup probíhá tak, že algoritmus smí mít vždy právě jeden ukazatel na vrchol
stromu, který na počátku každého přístupu ukazuje na kořen stromu. Dále v
každém kroku smí provést právě jeden z kroků tak, jak byly popsány v úvodu této kapitoly, případně rotaci hrany mezi aktuálním vrcholem a jeho rodičem. 

Řekneme, že čas běhu algoritmu je počet těchto kroků, které za sekvenci
přístupů provede, plus jedna\footnote{Čas strávený výpočtem toho, jaký krok se má provést, zanedbáme.}. O vrcholu stromu řekneme, že jsme se ho při daném
přístupu \emph{dotkli}, pokud na něj někdy během tohoto přístupu ukazoval
ukazatel algoritmu.  \end{definice}

Takovému přístupovému algoritmu se někdy také říká \emph{offline přístupový
algoritmus}. V praxi ale potřebujeme přístupy provádět online.

\begin{definice}
\emph{Online přístupový algoritmus} je takový přístupový algoritmus, jehož
rozhodnutí během $i$-tého přístupu nijak neovlivňují hodnoty $x_j$ z přístupové
posloupnosti pro $j>i$. Na druhou stranu si tento algoritmus smí v každém
vrcholu uložit až $\mathcal O(1)$ slov paměti informací (nikoli však ukazatele
na vrcholy).
\end{definice}

Všimneme si, že běžné algoritmy binárních vyhledávacích stromů tuto definici
splňují -- Například červenočerné a AVL stromy potřebují v každém vrcholu
jediný bit informace, splay strom se obejde zcela bez dalších informací.

Pro danou přístupovou sekvenci $X$ existuje přístupový algoritmus, který ji
vykoná optimálně, tedy v nejkratším čase ze všech možných algoritmů. Tento
počet kroků označíme $\opt(X)$. Zde předpokládáme, že je strom na začátku v
nejlepší možné konfiguraci. Tím však nesnížíme potřebný čas na přístupy o více
než aditivní $\mathcal O(n)$, protože z libovolného BVS je možné pomocí
$\mathcal O(n)$ rotací vytvořit libovolný jiný (nad tou samou množinou klíčů).
Proto budeme dále zkoumat pouze přístupové posloupnosti $S$ takové, že $|S| \in
\Omega(n)$. Vzhledem k tomu, že nahlédneme, že $\opt(x)\geq |S|$, je tento
faktor asymptoticky zanedbatelný. 


\begin{definice}
O přístupovém algoritmu řekneme, že je \emph{$f(n)$-kompetitivní}, pokud každou
posloupnost přístupů $X$ nad univerzem velikosti $n$ vykoná v čase $\mathcal
f(n)\cdot\opt(X)$. O online přístupovém algoritmu řekneme, že je
\emph{dynamicky optimální}, pokud je $\mathcal O(1)$-kompetitivní.
\end{definice}

Předtím, než si popíšeme některé konkrétní stromy, už pouze podotkneme, že
existence dynamicky optimálního algoritmu je otevřeným problémem. Na druhou
stranu například o splay stromech vyslovili \citet{splay} hypotézu, že jsou
dynamicky optimální.


\section{Meze optimality}

\section{Splay stromy}

\section{Tango stromy}

\section{Multisplay stromy}

\section{Geometrický náhled na stromové operace}
