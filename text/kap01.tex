\chapter{Teoretický úvod}

V této kapitole zavedeme formální definici binárního vyhledávacího stromu a
pojednáme o jeho výhodách oproti jiným způsobům uložení dat. Dále přiblížíme
(staticky) optimální strom. Pro dynamické stromy připomeneme některé běžné
vyvažovací strategie a také představíme jiné, novější přístupy například třídu
rankově vyvážených stromů a jejich speciální případ Weak AVL strom. Potom
představíme několik mezí optimality, vlastností, které musí mít optimální
binární vyhledávací strom. Nakonec se podíváme na Tango stromy a Multisplay
stromy a dokážeme, že tyto meze splňují (nebo, v případě tango stromu, že je
téměř splňují).

\section{Binární vyhledávací strom}
\def\U{\mathcal U}
\def\o{\mathcal O}

Častým úkolem datové struktury je udržovat informaci o množině prvků. Formálně mějme univerzum $\mathcal U$, lineární uspořádání na tomto univerzu $\geq$, a jeho podmonožinu $M$. Budeme chtít datovou strukturu, která nám bude umět o každé prvku $\mathcal U$ říct, jestli náleží množině $M$. Po takových strukturách často chceme nějaké z následujících operací:
\begin{itemize}
\item pro prvek $x \in \U$ rozhodnout, zda $x\in M$,
\item přidat prvek $x \in \U$ do množiny $M$,
\item odstranit prvek $x\in M$ z $M$,
\item vyjmenovat všechny prvky $M$,
\item pro daný prvek $x\in \U \setminus M$ najít nejbližší větší případně menší prvek $M$,
\item pro daný prvek $x\in M$ najít předchůdce, případně následovníka,
\item pro dané dva prvky $x,y\in\U$ takové, že $x<y$, vyjmenovat všechny prvky $z\in M$ takové, že $x \leq z \leq y$.
\end{itemize}

Volba struktury v každém konkrétním případě závisí na tom, které z těchto operací potřebujeme v naší konkrétní aplikaci podporovat. Pokud nám například stačí umět prvky vyjmenovat, postačí nám obyčejné pole. Pokud chceme umět vyhledávat, přidávat i mazat v konstantním čase, je nejlepší volbou nějaká varianta hashovací tabulky. Zejména poslední tři zmíněné operace ale hashovací tabulka umí pouze v čase $\Theta(|M|)$. Pokud tedy tento typ dotazů potřebujeme struktuře pokládat, můžeme využít nějakou ze stromových datových struktur, které v této práci popíšeme.

\begin{definice}
Mějme datovou strukturu, která se skládá z vrcholů. Každý vrchol může mít až
dva následníky, kterým budeme říkat \emph{levý} a \emph{pravý syn}. Jeden
vrchol nazveme \emph{kořen}. Všechny ostatní vrcholy pak budou mít právě
jednoho \emph{rodiče} (tak, že pokud vrchol $u$ je rodičem vrcholu $v$, pak
vrchol $v$ je pravým nebo levý synem vrcholu $u$)\footnote{V některých implementacích si vrcholy nemusí pamatovat ukazatele na své rodiče, my budeme však prozatím předpokládat, že si je pamatují.}. Pro vrchol $v$ nazveme
\emph{podstrom vrcholu $v$} množinu vrcholů, do nichž se dá dostat z z vrcholu
$v$ libovolnou\footnote{Posloupnost může být i prázdná.} posloupností operací \uv{přejdi do
pravého syna} a \uv{přejdi do levého syna}.

V každém vrcholu bude pak právě jedna hodnota, a navíc bude pro vrchol $v$ platit, že hodnoty ve všech vrcholech v podstromu levého syna $v$ jsou ostře menší než hodnota ve $v$, a hodnoty v podstromu pravého syna naopak ostře větší.

Nakonec si zapamatujeme ukazatel na kořen tak, abychom ho kdykoli v konstantním čase nalezli.

Této datové struktuře budeme říkat \emph{Binární vyhledávací strom}
\end{definice}

Při práci se stromem budeme předpokládat, že následující operace umíme v konstantním čase:
\begin{itemize}
\item Nalezni kořen,
\item přejdi do pravého, případně levého syna aktuálního vrcholu,
\item přejdi do rodiče aktuálního vrcholu,
\item přečti či zapiš hodnotu v aktuálním vrcholu,
\item Pokud aktuální vrchol nemá pravého či levého syna, vytvoř na jeho místě nový vrchol s danou hodnotou.
\end{itemize}
Vyhledávání hodnoty $x$ ve stromu pak můžeme provést podle následujícího algoritmu:
\begin{enumerate}
\item Nalezni kořen,
\item podívej se, zda je hodnota $x$ rovna hodnotě v aktuálním vrcholu, a pokud ano, nahlaš nález,
\item je-li hodnota $x$ větší než hodnota v aktuálním vrcholu, podívej se, zda existuje pravý syn aktuálního vrcholu. Pokud ano, přejdi do něj a vrať se ke 2. kroku algoritmu, jinak ohlaš, že  $x$ ve stromu není.
\item je-li hodnota $x$ menší než hodnota v aktuálním vrcholu, podívej se, zda existuje levý syn aktuálního vrcholu. Pokud ano, přejdi do něj a vrať se ke 2. kroku algoritmu, jinak ohlaš, že  $x$ ve stromu není.
\end{enumerate}

Podobným algoritmem budeme prvky přidávat, pouze místo nahlášení neexistence na místě chybějícího syna založíme nový vrchol. Mazání prvku provedeme následovně:
\begin{itemize}
\item Je-li mazaný prvek list, prostě ho smažeme,
\item má-li mazaný prvek právě jednoho syna, prvek smažeme a jeho syna připojíme pod rodiče mazaného prvku na na místo mazaného prvku,
\item má-li mazaný prvek dva syny, najdeme ve stromě prvek nejblíže větší, jeho
hodnotu zapíšeme na místo odstraňovaného prvku a odstraníme tento nejblíže větší
prvek, kterému nutně chybí přinejmenším levý syn (kdyby existoval, hodnota v
něm by byla menší než hodnota jeho rodiče, ale větší než hodnota odstraňovaného
vrcholu -- tedy jeho rodič by nemohl být nejblíže větší).  
\end{itemize} 

\section{Statická optimalita}

Přirozená otázka zní, jak vypadá binární vyhledávací strom, který pro danou
posloupnost přístupů vykoná nejméně operací. Pro začátek se omezíme na statické
stromy. To znamená stromy, které vybudujeme nad předem danou množinou klíčů, a
za běhu už nebudeme podporovat vkládání a mazání, ani neumožníme měnit
strukturu stromu. Na druhou stranu budeme předpokládat, že předem známe četnost
přístupů k jednotlivým prvkům. V tomto modelu se počet operací přesně rovná
počtu navštívených vrcholů.

Problém konstrukce optimálního stromu v tomto modelu vyřešil dynamickým
programováním \citet{staticoptimality}. Jeho algoritmus zde představíme.

Mějme množinu klíčů, pro jednoduchost {\tt [n]}\footnote{Tímto značení myslíme
množinu přirozených čísel od 1 do $n$ včetně.} a pole vah {\tt w}. Váha vrcholu
{\tt i} {\tt w[i]} může být buď pravděpodobnost, že daný přístup bude mít za
cíl naleznout vrchol {\tt i}, nebo počet výskytů {\tt i} v přístupové
posloupnosti. V prvním případě dostaneme jako váhu výsledného stromu střední
hodnotu počtu operací při přístupu, v druhém případě součet počtu operací přes
celou přístupovou posloupnost.

\code{
\l GetStaticallyOptimalTreeWeight(1, n, w):
\block
\l TreeWeights = Array[n + 1, n + 1]
\l \# V buňce i, j bude váha optimálního stromu 
\l \# na vrcholech i až i+j-1
\l WeightSums = Array[n + 1, n + 1]
\l \# V buňce i, j bude součet vah i-tého až j-1. vrcholu
\l for i in range(1, n + 2)
\block
\l for j in range(1, n + 2)
\block
\l WeightSums[i, j] = sum(w[i:j])
\endblock
\endblock
\l for Size in range(1, n + 1)
\block
\l for Start in range(1, n + 2 - Size)
\l End = Start + Size - 1
\block
\l Weight = $\infty$
\l for Root in range(start, start + size)
\block
\l LeftSubtreeWeight = \b TreeWeights[Start, Root - 1 - Start] \b + WeightSums[Start, Root]
\l RightSubtreeWeight = \b TreeWeights[Root + 1, End - Root] \b + WeightSums[Root + 1, End + 1]
\l Weight = min(Weight, w[Root] + LeftSubtreeWeight \b + RightSubtreeWeight) \refline{prirazovanivah}
\endblock
\l TreeWeights[Start, Size] = Weight
\endblock
\endblock
\l return TreeWeights[1, n]
\endblock
}

Kdyby nás zajímala nejen váha optimálního stromu, ale chtěli bychom ho skutečně
zkonstruovat, bylo by nutné na řádku \codelineref{prirazovanivah} a
následujícím ukládat nejen minimum, ale i tvar stromu na vrcholech {\tt Start}
až {\tt End}jako tvar optimálního levého a pravého podstromu s vrcholem {\tt
Root} jako kořenem. Výpočet obsahu pole {\tt WeightSums} by samozřejmě šel
udělat i kvadraticky s využitím prefixových součtů, nebo bychom ho vůbec
nemuseli počítat a mohli bychom si nechat pouze pole prefixových součtů.
Protože ale algoritmus má tak jako tak kvadratickou prostorovou a kubickou
časovou složitost, není z asymptotického pohledu nutné tuto část algoritmu
optimalizovat. Důkaz korektnosti algoritmu je založený na myšlence, že je-li v
optimálním stromě podstrom na vrcholech {\tt i} až {\tt j}, musí být tento
podstrom sám o sobě optimálním stromem na příslušných vrcholech. Potom tedy
stačí vyzkoušet všechny možné pozice kořene a rekurzivně spočítat jeho
optimální pravý a levý podstrom. Protože přímočaré aplikování tohoto postupu by
vedlo k tomu, že bychom cenu každého konkrétního podstromu počítali mnohokrát a
výsledný algoritmus by byl exponenciální, použijeme princip dynamického
programování a strom rekurze budeme procházet \uv{ze spoda}, což nám umožní
mezivýsledky ukládat a každý počítat pouze jednou.   

\section{Vyvažované stromy}

Co když však potřebujeme umět do struktury i vkládat? Můžeme pokračovat naivně
podle algoritmu uvedeného výše v této kapitole. \citet{searchingsorting}
ukázal, že pokud budeme prvky do stromu vkládat v náhodném pořadí\footnote{Bez
újmy na obecnosti předpokládáme, že do struktury postupně vložíme všechny prvky
z množiny $[n]$ pro vhodně zvolené $n$. Potom náhodné pořadí znamená pořadí
určené rovnoměrně náhodně zvolenou permutací z množiny $S_n$}, s vysokou
pravděpodobností dostaneme strom, v němž bude průměrná hloubka vrcholu $2 \ln n$.
\citet{Robson} dále ukázal, že pokud $H(n)$ střední hodnota hloubky nejhlubšího 
vrcholu ve stromě o $n$ vrcholech, pak $$\lim_{m\rightarrow
\infty}\frac{H(n)}{\ln(n)}\leq\alpha,$$ kde $\alpha$ je přibližně
$4{,}31107\dots$, přesně se jedná o největší kořen rovnice $$\alpha\cdot \ln
\frac{2e}{\alpha} = 1.$$ \citet{devroye} později dokázal, že výše uvedená
nerovnost je ve skutečnosti rovnost.

V praxi však často klíče potřebujeme vkládat i v pořadí, které není náhodné,
ale obsahuje nějaký vzorec. Například pokud bychom do (na počátku prázdného)
nevyvažovaného binárního stromu vložili čísla z množiny $[n]$ vzestupně, strom
zdegeneruje v jednu jedinou cestu, na níž bude njhlubší vrchol v hloubce $n$ a
průměrná hloubka vrcholu bude $(n+1)/2$. Takovýmto případům bychom se rádi
vyhnuli, proto prozkoumáme vyvažované stromy. Protože však všechny stromy,
které budeme zkoumat, budou vyvažované pomocí rotací hran, nejprve představíme,
co taková rotace hrany je.

\begin{definice}
Mějme vrchol BVS $u$ a jeho (bez újmy na obecnosti levého) syna $v$. Dále si
označíme $p$ rodiče vrcholu $u$, $A$ a $B$ popořadě levý a pravý podstrom $v$ a
$C$ pravý podstrom $u$. Pak \emph{(jednoduchá) rotace hrany $uv$} je operace,
při níž nastavíme vrchol $v$ jako syna $p$ (na stejné straně, kde byl původně
vrchol $u$), vrchol $u$ jako pravého syna $v$ a podstrom $B$ jako levý podstrom
$u$. Podstromy $A$ a $C$ zůstanou připojené k vrcholům $v$ a $u$ tak, jak byly.

Dále mějme vrchol $u$, jeho (bez újmy na obecnosti levého) syna $v$ a syna
vrcholu $v$, kterého si označíme $w$. Potom pokud $w$ je pravý syn, můžeme
provést \emph{dvojitou rotaci hran $wv$ a $vu$ typu zigzag}, která probíhá tak,
že nejprve zrotujeme hranu $wv$, a potom hranu $vu$. Pokud je $w$ levý syn,
můžeme provést \emph{dvojitou rotaci hran $wv$ a $vu$ typu zigzig}, která
probíhá tak, že nejprve zrotujeme hranu $vu$ a potom hranu $wv$.
\end{definice}

Dvojité rotace typu zigzig budeme potřebovat později v této kapitole. V této
sekci si vystačíme s jednoduchými rotacemi a dvojitými rotacemi typu zigzag,
proto budeme-li mluvit o dvojitých rotacích, budeme tím mít na mysli právě typ
zigzag.

\subsection{AVL stromy}

AVL strom poprvé představili \citet{AVL}. Jedná se o typ vyvažovaného stromu. V
AVL stromu platí, že hloubka pravého a levého podstromu každého vrcholu se liší
nejvýše o jedna. To znamená, že pokud $D(n)$ označíme minimální počet vrcholů
ve stromu hloubky $n$, dostáváme, že $D(0)=0$, $D(1)=1$ a
$D(i)=1+D(i-1)+D(i-2)$ pro každé $i$ větší než jedna. Pokud vyřešíme tuto
rekurenci, zjistíme, že $D(i)=\log_\varphi(i) + \o(1)$, kde $\varphi$ je zlatý
řez.

V každém vrcholu si budeme pamatovat navíc informaci o vyvážení tohoto vrcholu.
Tato informac může nabývat tří možných hodnot -- daný vrchol může být buď
\emph{vyvážený} (oba jeho podstromy jsou stejně hluboké), nebo \emph{nakloněný
doprava}, (jeho pravý podstrom je o jedna hlubší než levý), nebo
\emph{nakloněný doleva}\footnote{Vystačili bychom si i s jediným bitem, který by udával, zda je podstrom daného vrcholu hlubší než podstrom jeho souseda.}.

Vyhledání prvku probíhá stejně, jako v nevyvažovaném binárním vyhledávacím stromě. 
Vkládání je ale trochu komplikovanější. Následující pseudokód předpokládá neprázdný
AVL strom -- pro prázdný strom stačí vytvořit nový vrchol, nastavit ho jako vyvážený,


\code{
\l AVLinsert(Root, Element)
\block
\l Current = Root
\l Next = Current.value > Element ? Current.leftSon : Current.rightSon
\l while Next != NULL
\block
\l Current = Next
\l Next = Current.value > Element ? Current.leftSon : Current.rightSon
\endblock
\l New = Node(Element)
\l New.balance = VYVÁŽENÝ
\l New.parent = Current
\l if Current.value > Element
\block
\l Current.leftSon = New
\endblock
\l else
\block
\l Current.rightSon = New
\endblock
\l
\l Current = New
\l Next = Current.parent
\l while Next != NULL
\block
\l ComingFromRight = Current == Next.rightSon
\l if Next.balance == VYVÁŽENÝ
\block
\l Next.balance = ComingFromRight ? NAKLONĚNÝ DOPRAVA : NAKLONĚNÝ DOLEVA
\endblock
\l else if (Next.balance == NAKLONĚNÝ DOLEVA) == ComingFromRight
\block
\l Next.balance = VYVÁŽENÝ
\l return
\endblock
\l else
\block
\l
\l TADY JEŠTĚ KUS CHYBÍ
\l
\endblock
\l Current = Next
\l Next = Cureent.parent
}




\section{Výpočetní model}
Chceme-li mluvit o optimálním stromu, musíme nejprve specifikovat, co přesně
budeme za binární vyhledávací strom považovat. Definici, kterou zde
představíme, používali implicitně už \citet{splay}, formalizoval ji však až
\citet{tango}.



\begin{definice}
Mějme statické \emph{univerzum} klíčů $\mathcal U = \{1,2,\dots,n\}$. Dále
mějme binární strom nad těmito klíči takový, že pro každé dva vrcholy $v_1$,
$v_2$ platí, že je-li vrchol $v_1$ v podstromu vrcholu $v_2$, pak je v jeho
levém podstromu, právě když je jeho klíč menší než líč vrcholu $v_2$. Nakonec
mějme \emph{přístupovou posloupnost} $x_1,x_2,\dots,x_m$, kde $\forall i \in
\mathbb N, i\leq m$ platí $x_i\in \mathcal U$. Pak \emph{přístupový algoritmus
binárního vyhledávacího stromu} je algoritmus, který postupně provede přístupy
ke vrcholům s klíči $x_1, x_2,\dots,x_m$.

Přístup probíhá tak, že algoritmus smí mít vždy právě jeden ukazatel na vrchol
stromu, který na počátku každého přístupu ukazuje na kořen stromu. Dále v
každém kroku smí provést právě jednu z následujících operací: 
\begin{itemize}
\item Přesunout ukazatel na levého syna aktuálního vrcholu,
\item přesunout ukazatel na pravého syna aktuálního vrcholu,
\item přesunout ukazatel na rodiče aktuálního vrcholu,
\item provést rotaci hrany mezi aktuálním vrcholem a jeho rodičem.
\end{itemize}

Řekneme, že čas běhu algoritmu je počet těchto operací, které za sekvenci
přístupů provede, plus jedna. O vrcholu stromu řekneme, že jsme se ho při daném
přístupu \emph{dotkli}, pokud na něj někdy během tohoto přístupu ukazoval
ukazatel algoritmu.  \end{definice}

Takovému přístupovému algoritmu se někdy také říká \emph{offline přístupový
algoritmus}. V praxi ale potřebujeme přístupy provádět online.

\begin{definice}
\emph{Online přístupový algoritmus} je tekový přístupový algoritmus, jehož
rozhodnutí během $i$-tého přístupu nijak neovlivňují hodnoty $x_j$ z přístupové
posloupnosti pro $j>i$. Na druhou stranu si tento algoritmus smí v každém
vrcholu uložit až $\mathcal O(1)$ slov paměti informací (nikoli však ukazatele
na vrcholy).  
\end{definice}

Všimneme si, že běžné algoritmy binárních vyhledávacích stromů tuto definici
splňují -- Například červenočerné a AVL stromy potřebují v každém vrcholu
jediný bit informace, splay strom se obejde zcela bez dalších informací.

Pro danou přístupovou sekvenci $X$ existuje přístupový algoritmus, který ji
vykoná optimálně, tedy v nejkratším čase ze všech možných algoritmů. Tento
počet operací označíme $\opt(X)$. Zde předpokládáme, že je strom na začátku v
nejlepší možné konfiguraci. Tím však nesnížíme potřebný čas na přístupy o více
než aditivní $\mathcal O(n)$, protože z libovolného BVS je možné pomocí
$\mathcal O(n)$ rotací vytvořit libovolný jiný (nad tou samou množinou klíčů).
Proto budeme dále zkoumat pouze přístupové posloupnosti $X$ takové, že $|X| \in
\Omega(n)$. Vzhledem k tomu, že nahlédneme, že $\opt(x)\geq |X|$, je tento
faktor asymptoticky zandebatelný. 


\begin{definice}
O přístupovém algoritmu řekneme, že je \emph{$f(n)$-kompetitivní}, pokud každou
posloupnost přístupů $X$ nad univerzem velikosti $n$ vykoná v čase $\mathcal
f(n)\cdot\opt(X)$. O online přístupovém algoritmu řekneme, že je
\emph{dynamicky optimální}, pokud je $\mathcal O(1)$-kompetitivní.
\end{definice}

Předtím, než si popíšeme některé konkrétní stromy, už pouze podotkneme, že
existence dynamicky optimálního algoritmu je otevřeným problémem. Na druhou
stranu například o splay stromech vyslovili \citet{splay} hypotézu, že jsou
dynamicky optimální.

\section{Rankově vyvážené stromy}

\section{Meze optimality}

\section{Tango stromy}

\section{Multisplay stromy}
