%%%%%%%%%%%%%%%%%%%%%%%%%%%%%%%%%%%%%%%%
% Beamer Presentation
% LaTeX Template
% Version 1.0 (10/11/12)
%
% This template has been downloaded from:
% http://www.LaTeXTemplates.com
%
% License:
% CC BY-NC-SA 3.0 (http://creativecommons.org/licenses/by-nc-sa/3.0/)
%
%%%%%%%%%%%%%%%%%%%%%%%%%%%%%%%%%%%%%%%%%

%----------------------------------------------------------------------------------------
%   PACKAGES AND THEMES
%----------------------------------------------------------------------------------------

\documentclass{beamer}

\mode<presentation> {

% The Beamer class comes with a number of default slide themes
% which change the colors and layouts of slides. Below this is a list
% of all the themes, uncomment each in turn to see what they look like.

%\usetheme{default}
%\usetheme{AnnArbor}
%\usetheme{Antibes}
%\usetheme{Bergen}
%\usetheme{Berkeley}
%\usetheme{Berlin}
%\usetheme{Boadilla}
%\usetheme{CambridgeUS}
%\usetheme{Copenhagen}
%\usetheme{Darmstadt}
%\usetheme{Dresden}
%\usetheme{Frankfurt}
%\usetheme{Goettingen}
%\usetheme{Hannover}
%\usetheme{Ilmenau}
%\usetheme{JuanLesPins}
%\usetheme{Luebeck}
\usetheme{Madrid}
%\usetheme{Malmoe}
%\usetheme{Marburg}
%\usetheme{Montpellier}
%\usetheme{PaloAlto}
%\usetheme{Pittsburgh}
%\usetheme{Rochester}
%\usetheme{Singapore}
%\usetheme{Szeged}
%\usetheme{Warsaw}

% As well as themes, the Beamer class has a number of color themes
% for any slide theme. Uncomment each of these in turn to see how it
% changes the colors of your current slide theme.

%\usecolortheme{albatross}
%\usecolortheme{beaver}
%\usecolortheme{beetle}
%\usecolortheme{crane}
%\usecolortheme{dolphin}
%\usecolortheme{dove}
%\usecolortheme{fly}
%\usecolortheme{lily}
%\usecolortheme{orchid}
%\usecolortheme{rose}
%\usecolortheme{seagull}
%\usecolortheme{seahorse}
%\usecolortheme{whale}
%\usecolortheme{wolverine}

%\setbeamertemplate{footline} % To remove the footer line in all slides uncomment this line
%\setbeamertemplate{footline}[page number] % To replace the footer line in all slides with a simple slide count uncomment this line

%\setbeamertemplate{navigation symbols}{} % To remove the navigation symbols from the bottom of all slides uncomment this line
}
\usepackage[utf8]{inputenc}
\usepackage[czech]{babel}
\usepackage[IL2]{fontenc}
\usepackage{graphicx} % Allows including images
\usepackage{booktabs} % Allows the use of \toprule, \midrule and \bottomrule in tables

%----------------------------------------------------------------------------------------
%   TITLE PAGE
%----------------------------------------------------------------------------------------

\title[Dynamické vlastnosti stromů]{Dynamické vlastnosti stromů} % The short title appears at the bottom of every slide, the full title is only on the title page

\author{Viktor Němeček} % Your name
\institute[MFF UK] % Your institution as it will appear on the bottom of every slide, may be shorthand to save space
{
 \\ % Your institution for the title page
\medskip
\textit{} % Your email address
}
\date{2. 2. 2022} % Date, can be changed to a custom date

\begin{document}

\begin{frame}
\titlepage % Print the title page as the first slide
\end{frame}

%\begin{frame}
%\frametitle{Overview} % Table of contents slide, comment this block out to remove it
%\tableofcontents % Throughout your presentation, if you choose to use \section{} and \subsection{} commands, these will automatically be printed on this slide as an overview of your presentation
%\end{frame}

%----------------------------------------------------------------------------------------
%   PRESENTATION SLIDES
%----------------------------------------------------------------------------------------

%------------------------------------------------
\section{First Section} % Sections can be created in order to organize your presentation into discrete blocks, all sections and subsections are automatically printed in the table of contents as an overview of the talk
%------------------------------------------------

\subsection{Subsection Example} % A subsection can be created just before a set of slides with a common theme to further break down your presentation into chunks


%------------------------------------------------

\begin{frame}
\frametitle{Optimalita}
\begin{itemize}
\item Statická optimalita
\item Model dynamického BVS
\item Optimální strom
\begin{itemize}
\item Vidí do budoucnosti
\item Vykoná každou přístupovou posloupnost s nejmenším možným počtem navštívených vrcholů
\end{itemize}
\item Online přístupový algoritmus
\begin{itemize}
\item Nevidí do budoucnosti
\item Má ve vrcholech $\mathcal O(1)$ slov paměti pomocné informace
\end{itemize}
\item Odhady
\begin{itemize}
\item Interleave bound
\end{itemize}
\end{itemize}
\end{frame}

\begin{frame}
\frametitle{Stromy}
\begin{itemize}
\item Červenočerný strom
\item Splay strom
\item Tango strom
\item Multisplay strom
\end{itemize}
\end{frame}


\begin{frame}
\frametitle{Splay strom}
\begin{itemize}
\item Staticky optimální
\item Inorder průchod v $\mathcal O(n)$
\item Dynamická optimalita -- otevřený problém
\end{itemize}
\end{frame}

\begin{frame}
\frametitle{Tango strom}


\begin{figure}[h!]

  \centering
  \includegraphics[width=.9\linewidth]{tango}
\caption{Tango strom -- referenční strom a odpovídající \uv{reálný} strom} 

\label{obr:cut_tango} 
 
\end{figure}
\end{frame}


\begin{frame}
\frametitle{Tango strom}
\begin{itemize}
\item Tvořen pomocnými červenočernými stromy
\item $\log\log(n)$-kompetitivní
\item Náhodný přístup v $\mathcal O(\log n \cdot \log\log n)$
\end{itemize}
\end{frame}

\begin{frame}
\frametitle{Multisplay strom}
\begin{itemize}
\item Jako tango strom, ale pomocné stromy jsou splay stromy
\item Teoreticky spojuje to nejlepší z Tango stromu a Splay stromu
\begin{itemize}
\item $\log\log(n)$-kompetitivní
\item Inorder průchod v $\mathcal O(n)$
\item Amortizovaně $\mathcal O(\log n)$ na přístup
\item Worst case $\mathcal O(\log^2 n)$ na přístup
\end{itemize}
\end{itemize}
\end{frame}

\begin{frame}
\frametitle{Výsledky -- Sekvenční přístupová posloupnost}
\begin{figure}[h!]
  \centering
  \includegraphics[width=.75\linewidth]{../text/graphs/touch_s}
\caption{Počet dotyků -- sekvenční přístupová posloupnost} 
\end{figure}
\end{frame}

\begin{frame}
\frametitle{Výsledky -- Sekvenční přístupová posloupnost}
\begin{figure}[h!]
  \centering
  \includegraphics[width=.75\linewidth]{../text/graphs/time_s}
\caption{Čas na přístup -- sekvenční přístupová posloupnost} 
\end{figure}
\end{frame}

\begin{frame}
\frametitle{Výsledky -- náhodná přístupová posloupnost}
\begin{figure}[h!]
  \centering
  \includegraphics[width=.75\linewidth]{../text/graphs/time_r}
\caption{Čas na přístup -- náhodná přístupová posloupnost} 
\end{figure}
\end{frame}

\begin{frame}
\frametitle{Výsledky -- náhodná přístupová posloupnost}
\begin{figure}[h!]
  \centering
  \includegraphics[width=.75\linewidth]{../text/graphs/randomized_rb}
\caption{Červenočerný strom -- náhodná a sekvenční alokace} 
\end{figure}
\end{frame}

\begin{frame}
\frametitle{Výsledky -- sekvenční přístupová posloupnost}
\begin{figure}[h!]
  \centering
  \includegraphics[width=.75\linewidth]{../text/graphs/randomized_splay}
\caption{Splay strom -- náhodná a sekvenční alokace} 
\end{figure}
\end{frame}


\begin{frame}
\frametitle{Výsledky -- další zkoumané posloupnosti}
\begin{itemize}

\item Bit reversal posloupnost na páteři
\begin{itemize}
\item Umělá, vyrobená přímo na míru tango stromu
\end{itemize}

\item Sekvenční podmnožinová posloupnost
\begin{itemize}
\item Nejlepšího výsledku dosáhl splay strom, stále ale výrazně horší než pro \uv{normální} podmnožinovou posloupnost
\end{itemize}

\item Náhodná podmnožinová posloupnost
\begin{itemize}
\item Nejlepšího výsledku dosáhl splay strom, pokud byla velikost podmnožiny $<n/100$. Červenočerný strom se naopak pro rostoucí podmnožinu zrychloval.
\end{itemize}

\item Proměnlivě podmnožinová posloupnost
\begin{itemize}
\item Nejlepšího výsledku dosáhl pro většinu voleb velikosti podmnožiny červnočerný strom.
\end{itemize}

\end{itemize}
\end{frame}


\begin{frame}
\frametitle{Proměnlivě podmnožinová přístupová posloupnost}
\begin{figure}[h!]
  \centering
  \includegraphics[width=.75\linewidth]{../text/graphs/comparison_e}
\caption{Čas na přístup -- $n=10^7$} 
\end{figure}
\end{frame}

\begin{frame}
\Huge{\centerline{Děkuji za pozornost}}
\end{frame}

%----------------------------------------------------------------------------------------

\end{document}
